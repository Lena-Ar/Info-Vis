\documentclass[usegeometry=true]{scrartcl}
\usepackage[ngerman]{babel}
\usepackage[T1]{fontenc}
\usepackage{lmodern}
\usepackage[utf8]{inputenc}
\usepackage{hyperref}
\usepackage{amssymb}
% Dimensionen bitte nicht ändern. 
\usepackage[left=2cm, right=2cm, top=2cm, bottom=2cm, bindingoffset=1cm, includeheadfoot]{geometry}
%Zeilenabstand bitte nicht ändern
\usepackage[onehalfspacing]{setspace}

\usepackage[backend=biber,style=numeric,]{biblatex}\addbibresource{literatur.bib}

\begin{document}
% ----------------------------------------------------------------------------
\subject{Projektbericht zum Modul Information Retrieval und Visualisierung Sommersemester 2022}
\title{Marktanalyse des Videospielemarktes}
\subtitle{Analyse und Visualsierung der Verkaufszahlen der Videospiele auf der Platform XBoxOne}
\author{Lena Arloth}% obligatorisch
%\date{10.9.2022}
\maketitle% verwendet die zuvor gemachte Angaben zur Gestaltung eines Titels
% ----------------------------------------------------------------------------
\newpage
%Inhaltsverzeichnis:
\tableofcontents
% ----------------------------------------------------------------------------
% Gliederung und Text:
\newpage
\section{Einleitung}
Der weltweite Videospielmarkt über alle Plattformen hinweg ist mit einem Volumen von 156,17 Milliarde Euro im Jahre 2021 sehr groß und wird 2022 voraussichtlich auf 176,13 Milliarden Euro steigen.\cite{Statista.2022b}
Entsprechend viel Geld und Potenzial durch Wachstum verbirgt sich in ihm.
Gleichzeitig benötigt es für die Entwicklung neuer Videospiele hohe Investitionen von Seiten der Entwicklerstudios und Verleger, folgend Publisher genannt, die erst nach einer gewissen Sicherheit durch Analyse des Marktes getroffen werden sollten.
Um Prognosen und Tendenzen für die Zukunft finden zu können, braucht es die Erkenntnisse aus vergangenen Daten.
Eine Möglichkeit hierzu ist die Analyse der totalen Verkaufszahlen der Videospieltitel aufgeschlüsselt nach Genre weltweit.

Als Problemstellung ergibt sich das fehlende Wissen der Entscheider innerhalb der Publisher, in welche Videospieltitel in welchem Genre und mit welcher Regionsausrichtung sie künftig investieren sollten, um das Potenzial des großen und wachsenden Videospielmarktes möglichst gut zu nutzen.
Investition bezieht sich hierbei nicht zwingend auf eine konkrete Entscheidung für die Auftragsvergabe eines neuen Titels, sondern vor allem auf die Veranlassung detaillierter, umfassender Marktstudien.
Erst dann kann eine fundierte endgültige Entscheidung getroffen werden.
Es ergibt sich eine durchzuführende, grobe Marktanalyse der Verkaufszahlen der Videospielindustrie einer Plattform aus Sicht der Publisher.
Sie fungiert als Grundlage und Unterstützung weiterer Entscheidungen und Investitionen.

Die Relevanz ergibt sich aus zuvor beschriebenem und der Notwendigkeit von Marktanalysen als Voraussetzung für unternehmerische Entscheidungen, Chancen und Risiken sowie Potenialen.\cite{Fleig.2020}
Es ist für Unternehmen wie Publisher sehr wichtig zu erfahren, was und in welchem Genre künftig mit guten Verkaufschancen entwickelt werden soll.
Weiterhin ist es relevant, mittels einer Marktanalyse zu erfahren, wie das Unternehmen selbst im Markt aufgestellt ist und was die Konkurrenz bietet. 
Um dies zu erreichen, ist es wichtig, Muster, Cluster und Korrelationen zwischen den Verkaufszahlen verschiedener Regionen in einem Genre zu erkennen. 
Für ein erleichtertes Verständnis sowie das Hervorheben von Auffälligkeiten sind Techniken aus dem Bereich der Visual Analytics von kritischer Relevanz.

Aus zuvor beschriebenem leitet sich folgende Fragestellung ab, die es mittels einer groben Marktanalyse aufbereitet, verdeutlicht und sichtbar gemacht durch Visualisierungstechniken zu beantworten gilt:
In welches Genre und mit welchem Fokus auf die Regionen der Welt sollte ein Publisher künftig mit der Beauftragung detaillierter, teuerer Marktstudien zur Entscheidung über die Auftragsvergabe neuer Videospieltitel investieren?
%Tipps zu Latex und Koma-Script für Hausarbeiten sind im \href{http://mirrors.ctan.org/info/latex-refsheet/LaTeX_RefSheet.pdf}{LaTeX Reference Sheet for a thesis with KOMA-Script} von Marion Lammarsch und Elke Schubert zusammengefasst. 
%Der Bericht fällt in die Kategorie von InfoVis-Paper, die Tamara Munzner Design Study nennt \cite{Munzner2008}: In der Einleitung sollen sie zuerst das Zielproblem beschrieben. Daraus sollen sie Fragestellungen motivieren, die mittels Techniken der Informationsvisualisierung beantwortet werden können. 
%In dem Abschnitt direkt unter der Überschrift Einleitung sollen Sie nach einer kurzen Einleitung Fragestellungen und das Zielproblem motivieren und besschreiben. 

\subsection{Anwendungshintergrund}
Marktanalysen sind für Unternehmen wie Publisher von zentralem Wert, da sie mit ihnen strategische Zukunftsfragen zur Ausrichtung beantworten können. 
Sie müssen unter anderem Fragen zu Produktverbesserungen, -erweiterungen oder -neuerungen, Investitionen in diese und Potenziale und Risiken durch Korrelationen, Muster und Ausreißer beantworten können.\cite{Fleig.2020}
Dabei kann sie detaillierter oder oberflächlicher sein. In dieser Anwendung ist eine oberflächliche Analyse geplant, mittels derer Entscheidungen für detailliertere getroffen werden können.
Der Vorteil in diesem Vorgehen liegt in der Ersparnis von Zeit und Geld. 

Der Videospielmarkt ist in verschiedene Segmente geteilt, die wiederum unterteilt sind. 
Für diese Arbeit wird das Segement der Spieleplattform und Konsole \textit{XBoxOne} von Microsoft gewählt.
Dies liegt zum einen in der Verfügbarkeit der Daten begründet und zum anderen in der Beliebtheit der Konsole und der Konkurrenz zu anderen Plattformen wie \textit{Playstation 4} und \textit{Computer}.
Gleichzeitig ergeben Analysen innerhalb einer Plattform für die Entscheidungsunterstützung künftiger Investitionen in Titel dieser Plattform mehr Sinn, da die Plattformen untereinander nicht kompatibel sind.

Die Konsole \textit{XBoxOne} von Microsoft verkaufte sich besonders in den Jahren 2013 bis Anfang 2019 gut mit einer Spitze im Dezember 2015 mit rund 715.000 verkauften Einheiten im Monat.
Mittlerweile ist die Beliebtheit der Konsole zwar stark zurückgegangen, doch lassen sich aus den Verkaufszahlen von Videospielen ihrer Plattform weiterhin Erkenntnisse für die Zukunft und potenzielle Investitionen der Publisher ableiten.\cite{Statista.2022}
Hierbei geht es auch um Investitionen in neue Titel für die \textit{XBoxOne}, aber vor allem um solche in die Nachfolgekonsolen \textit{XBox Series X} und \textit{XBox Series S}. 
Durch die Abwärtskompatibilität scheint es wahrscheinlich, dass Nutzer der \textit{XBoxOne} auch die Nachfolgekonsolen nutzen.\cite{GamesWirtschaft.2021} 
Somit behalten die Verkaufszahlen der \textit{XBoxOne} ihre, wenn auch leicht verminderte, Aussagekraft.
Der Sinn einer überischtsartigen, grobgranularen Marktanalyse dieses Segments bleibt bestehen.
Zur besseren Vergleichbarkeit und Aussagekraft der Verkaufszahlen und ihrer Auffälligkeiten der Videospiele ist eine Filterung nach Genre sinnvoll.
%Sie müssen genug Hintergrund bereitstellen, so dass die Lesenden sich ein Urteil bilden können, ob ihre Lösung funktioniert. 
%Sie sollen die Lesenden jedoch nicht mit Anwendungsdetails so überschütten, dass der Fokus auf die Fragen zur Informationsvisualisierung untergehen. 

\subsection{Zielgruppe}
Die in dieser Arbeit betrachtete Zielgruppe setzt sich aus Entscheidern der Publisher von Videospielen zusammen. 
Diese Entscheider sind vorranging Personen des oberen Management, aber auch des mittleren Managements, vorrangig aus dem Bereich Marketing und Forschung und Entwicklung. 
Das mittlere Management übernimmt dabei neben dem Mitentscheiden auch die Aufgabe des Präsentierens der Analysen mittels Visualisierungen für das obere Management.
Weiterhin zählen in begrenztem Rahmen auch Stakeholder der Publisher zur Zielgruppe, da sie über Entscheidungen des Unternehmens zu künftigen Strategien informiert werden sollten. 

In der Zielgruppe existiert kein detailliertes bzw. stark spezifiziertes Vorwissen zu speziellen Visualisierungstechniken. 
Jedoch ist anzunehmen, dass sich vor allem das mittlere und obere Management mit in der Betrieswirtschaftslehre häufiger vorkommenden Visualisierungen auskennt und diese ohne viele Erklärungen auswerten kann.
Dazu zählen neben Balken- und Kreisdiagramme, Box Plots und Zeitreihendiagramme auch Scatterplots, explizite Baumdiagramme und Parallele Koordinaten Plots. 
Bei spezielleren Visualisierungstechniken wären mehr Erklärungen sowie mehr Zeit zur Analyse und Entscheidung nötig. 
Dies wäre für eine möglichst schnelle und kostengünstige Entscheidung zu detaillierteren Marktstudien, wobei noch keine konkrete Entscheidung für einen Titel getroffen wird, nicht zielführend.

Durch die Visualisierungen werden mehrere Informationsbefürfnisse adressiert. 
Zum einen wird das Bedürfnis nach übersichtsartigen Informationen zur Videospielindustrie im Sinne der eigenen Position und der der Konkurrenz für die \textit{XBoxOne} bezüglich der Verbindungen von Publisher, Genre und Videospiel adressiert.
Zum zweiten werden Informationsbefürfnisse nach Zusammenhängen und Mustern zwischen Regionen in den jeweiligen Genres befriedigt, um daraus strategische Entscheidungen ableiten zu können.
Zum dritten wird das Bedürfnis nach konkreten Verkaufszahlen der einzelnen Spieletitel angesprochen.
%Beschreiben sie die Personengruppe oder Personengruppen, die das von ihnen benannte Anwendungsproblem lösen möchte. Auf welches Vorwissen können sie in dieser Gruppen von Anwenderinnen aufbauen? 
%Welche Informations"-bedürf"-nisse werden durch die Visualisierungen adressiert?

\subsection{Überblick und Beiträge}
Die in dieser Arbeit verwendeten Daten stammen aus einem Datensatz für die \textit{XBoxOne} mit den Attributen Tabellenposition, Publisher, Jahr der Veröffentlichung, Genre, Verkaufszahlen global sowie in den Regionen Nordamerika, Europa, Japan und Rest der Welt von der Plattform Kaggle.\cite{SID_TWR.} 
Die Visualisierungen entwickeln sich vom Groben zum Detaillierten.
In einer ersten Visualisierung wird zur Unterstützung einer Übersicht über die Publisher, Genre und Videospieltitel ein hierarchisches, explizites Baumdiagramm verwendet. 
Dadurch wird zudem ein Ansatzpunkt zur Auswahl der für die Publisher zu untersuchenden Genres gegeben.
Zur Erkennung von Mustern und Ausreißern in den Verkaufszahlen eines Genres auch über alle Regionen hinweg bzw. in einer Ansicht aller Dimensionen gleichzeitig dient der folgende Parallele Koordinaten Plot.
Zuletzt können in der dritten Visualisierung, dem Scatterplot, Korrelationen zwischen zwei Regionen sowie Ausreißer genauer betrachtet werden.
In allen Visualisierungen sind Spieletitel mit Detailinformationen zu Publishern und Verkaufszahlen für einen konkreten Vergleich erkennbar.
Mittels aller Visualisierungen kann die übergeordnete Fragestellung beantwortet werden, welches Genre durch positive Korrelationen, Tendenzen zu wenig Konkurrenz oder hohen Spielerzahlen durch viele Titel in dem Genre und der Analyse von Mustern und Ausreißern Potenzial bietet und damit detaillierterer Marktanalysen bedarf.
%In diesem Abschnitt geben sie einen kurzen Überblick über die Daten und verwendeten Visualisierungen. Dann benennen sie die Beiträge ihres Projekts. Diese Beiträge müssen sie in den hinteren Teilen des Berichts genauer ausführen und belegen.

\section{Daten}
Der genutzte Datensatz \textit{"Video Games Sales Dataset"} stammt von der Plattform \textit{Kaggle} vom Nutzer \textit{SID\_TWR}.\cite{SID_TWR.} 
Die Daten entstanden laut SID\_TWR durch Erweiterung der Daten eines Web Scrapes von \textit{VGChartz Video Games Sales} motiviert durch Gregpry Smith um weitere Attribute aus einem Web Scrape von \textit{Metacrtitic}.  
Von den drei zur Verfügung gestellten Datensätzen, wird \textit{XBoxOne\_GameSales} für dieses Projekt ausgewählt. 
Die zuvor erwähnten Erweiterungen um Attribute stammend von \textit{Metacrtitics} sind hier nicht enthalten. 
Der Originaldatensatz liegt als CSV-Datei vor und beinhaltet zehn Spalten mit 613 einzelnen Positionen. 

Zu jeder \textit{Position} sind der Videospielname sowie das jeweilige Jahr der Veröffentlichung aufgelistet. 
\textit{Genre} kategorisiert die Videospiele entsprechend und \textit{Publisher} ordnet jedem Videospiel seinen Verleger zu. 
Die Attribute \textit{North America}, \textit{Europe}, \textit{Japan}, \textit{Rest of World} und \textit{Global} stellen die Verkäufe der Videospiele in \textit{millions of units}, also Millionen Stück verkaufter Kopien, in diesen Regionen dar. 
Der für das Projekt gewählte Datensatz bildet Videospiele von 2013 bis circa 2020 ab. 
Aufgrund der fehlenden Aktualisierung der Verkaufszahlen auf \textit{VGChartz Video Games Sales} ab dem Laufe des Jahres 2018, ist davon auszugehen, dass die für das Projekt verwendeten Daten kumulierte Verkaufszahlen ab 2013 bis circa 2018 abbilden. 

Die vorhandenen Daten werden als gut geeignet für die Zielgruppe und das zuvor eingeleitete Zielproblem eingeschätzt.  
Sie ermöglichen eine grobe Übersicht über die verschiedenen Publisher, ihre Videospielen und bedienten Genres. 
Weiterhin schlüsseln sie detailliert auf, wie sich ein Videospiel kumuliert seit seiner Veröffentlichung global, aber auch in den einzelnen Weltregionen verkaufte. 

\subsection{Bereitstellung und Vorverarbeitung der Daten}
Die Originaldaten werden als CSV-Datei heruntergeladen und zusätzlich zur Bearbeitung in \textit{Open Office Calc} in das ODS-Format überführt. 
Alle Daten sowie das Projekt und dieser Bericht werden in einem öffentlichen GitHub Repository bereitgestellt. 
Die Datendateien sind im Ordner \textit{Daten} in den Unterordnern \textit{CSV}, \textit{JSON} und \textit{Tabelle} je nach Format zu finden. 

Zur Erstellung des expliziten Baumdiagramms wird eine JSON-Datei benötigt, in der die Beziehungen der Publisher, Genre und Videospiele zueinander beschrieben sind. 
Videospiele seien Kinder der Genres, die wiederum Kinder der Publisher sind und unter einem Wurzelknoten zusammengefasst werden. 
Für die Realisierung des Scatterplots und des Parallelen Koordinaten Plots wird eine CSV-Datei benötigt. 
In dieser sind die nachfolgend detailliert beschriebenen Modifikationen enthalten. 

Der Datensatz enthält fehlende Werte bei den Publishern und den Verkaufszahlen jeder Region sowie für die Anwendung nicht benötigte Informationen. 
Zur besseren Nachvollziehbarkeit der Datenvorberarbeitung wird im GitHub Repository im Ordner \textit{Daten} eine zweite README.md-Datei erstellt, die den Fortlauf der Vorverarbeitung dokumentiert. 
Die benötigte Datengrundlage für vorliegende Arbeit wird in sieben Schritten erreicht, die folgend grob beschrieben sind und im Detail in angesprochener README nachvollzogen werden können.

Nach dem Herunterladen der Originaldateien und der Konvertierung dieser werden Testversionen der Originaldaten mit den ersten 20 Positionen im CSV- und ODS-Format erstellt.
Sie werden um die Spalte des Erscheinungsjahres reduziert.
Sie dienen der Übersichtlichkeit und Funktionalität in der Entwicklungsphase.

Es folgt die eigentliche Modifikation der Originaldaten für das Projekt.
Dabei werden neue Dateien erstellt, um transparent die Originaldaten weiterhin zur Verfügung stellen zu können.
Die Spalte des Erscheinungsjahres des jeweiligen Videospiels wird aus der hier durchgeführten Marktanalyse und ihrer Visualsierung ausgegliedert und gelöscht. 
Hieraus können weder Daten für ein Zeitreihendiagramm extrahiert werden, noch macht ein spezifischer Vergleich der Verkaufszahlen in Relation zum Erscheinungsjahr für die Problemstellung und deren Lösung in einem ersten Überblick Sinn. 
Es werden zusätzlich jene Positionen, also Videospiele, eliminiert, die einen Wert von Null in den Verkaufszahlen aller Regionen bzw. automatisch global aufweisen. 
Sie sind aufgrunddessen fehlerhaft und haben in der Visualisierung keine Verwendung. 
Zur Ausschließung von Fehlern bei der manuellen Vorverarbeitung sowie universelleren Einsetzbarkeit des Quellcodes und Änderung der Rohdaten wird dieser Schritt zusätzlich in Elm programmiert.
Aufgrund von Irrelevanz werden jene Positionen gelöscht, denen kein Publisher zugeordnet ist oder dessen Wert \textit{Unknown}, sprich unbekannt ist. 
Zusätzlich wird der Datensatz um die Positionen minimiert, die von einem Publisher stammen, der nur ein Videospiel verlegt. 
Es ist davon auszugehen, dass sich jene Publisher sehr spezifisch ausgerichtet haben und keine hier visualisierte überischtsartige Marktanalyse benötigen.
Weiterhin wird der Name des Publishers \textit{Namco Bandai Games} in \textit{Bandai Namco Games} vereinheitlicht, da sie durch Umbenennung 2014 denselben Verleger darstellen. 

Mittels des Online-Tools \textit{convertcsv.com} wird die Konvertiertung der zusätzlich erstellten CSV-Datei in eine JSON-Datei vorgenommen.
Diese wird auf Fehler in den Abhängigkeiten kontrolliert, der Wurzelknoten hinzugefügt sowie die entstandenen leeren Felder gelöscht. 
Zur Erzielung der gewünschten hierarchischen Struktur und Visualisierung wird pro Genre der jeweilige Eltern-Publisher wenn möglich sinnvoll abgekürzt in Klammern hinzugefügt. 
Auch hier wird eine Testdatei mit 20 Videospielen erstellt.

Eine Umrechnung der Einheiten der Verkäufe ist nicht ratsam, da sie schon übersichtlich vorliegt. 
Zur Erläuterung dieser dienen Informationstexte und die passenden Achsenbeschriftungen.

Die Herausfilterung der Positionen, die in nur einer Region Null-Werte in den Verkäufen aufzeigen wurde in Betracht gezogen, jedoch verworfen.
Solche Videospiele würden nicht angezeigt, obwohl auch Informationen über keine erfolgten Verkäufe und die Region, in der nichts verkauft werden konnte, relevant sind. 
Zu beachten ist jedoch, dass als Null-Werte auch solche aufgeführt sind, deren Verkaufszahlen bei unter 0.01 Millionen Stück verkaufter Videospielkopien liegen. 
%Beschreiben Sie vorhandenen Daten. Gehen sie kritisch darauf ein, in wie weit sich die Daten für die Bearbeitung der Fragestellungen und dem Erreichen von Lösungen für die oben beschriebene Zielgruppen eignen. Haben sie die Daten sinnvoll mit weiteren Datenquellen ergänzt? Wenn ja, wie?
%Erklären sie die technische Bereitstellung der Daten.
%Wie sind die Daten zugänglich? Welche Formate werden genutzt. Gibt es Besonderheiten beim Lesen der Formate?
%Beschreiben sie die Datenvorverarbeitung.
%Welche Datenvorverarbeitungsschritte sind notwendig? Beschreiben Sie die einzelnen Schritte und begründen sie sie, z.B. warum werden manche Daten weggelassen, über welche Mengen werden Durchschnitte berechnet, warum sind die so berechneten Werte aussagekräftiger als andere Werte. Wenn möglich sollen sie die Datenvorverarbeitung in Elm programmieren, so dass ihre Anwendung auf eine Änderung der Rohdaten reagieren kan. 

\section{Visualisierungen}
Im folgenden Kapitel wird eine Analyse der Anwendung zur Lösung des einleitend beschriebenen Zielproblems durchgeführt anhand derer Anforderungen an die Visualisierungen abgeleitet werden. 
Schließlich werden die Visualisierungen präsentiert.

\subsection{Analyse der Anwendungsaufgaben}
Das Zielproblem dieses Projekts ist eine überblicksartige, grobe Marktanalyse des Videospielemarktes mit Fokus auf den Verkaufszahlen der Plattform \textit{XBoxOne} aus Sicht der Publisher.
Ziel der Marktanalyse ist eine Übersicht über das Genre- und Videospielangebot des eigenen Verlagshaus sowie der Konkurrenz und Erkenntnisse über Zusammenhänge innerhalb von Genres in und zwischen verschiedenen Regionen der Welt durch Muster, Cluster und Korrelationen zu erhalten.
Übergeordnet soll mittels dieser Informationen eine fundierte Entscheidung über Investitionen in detailliertere, zielgerichtetere Marktstudien in bestimmte Genres oder Regionen getroffen werden können, um den Publisher mit der Beauftragung neuer Videospiele strategisch für die Zukunft aufzustellen. 
Weiterhin sollen mit den Erkenntnissen erste Vorentscheide und Vorbereitungen für diese Spieleinvestitionen getroffen werden können.

%Unterfragen der Problemstellung hier definiert und analysiert statt Einleitung
%weiterhin wichtig: Sprache Englisch!!
Welche Publisher bieten welche und wieviele Videospiele in welchem Genre an?
Gibt es Muster, Cluster, Ausreißer oder Korrelationen in den Verkaufszahlen zwischen den einzelnen Regionen der Welt oder gar über alle Regionen hinweg?

%aus Einleitung hier eingefügt
Durch eine schnell erstellte, intuitiv erklär-, präsentier- und verstehbare Visualisierung der grobgranularen Daten der Marktanalyse können im Publishingunternehmen erste vorab gültige Strategieentscheidungen bzgl. Investitionen in Titel eines Gernes getroffen und Maßnahmen vorbereitet werden.
Weiterhin kann dann eine detaillierte und deutlich aufwenigere und damit teurere Marktstudie in den zuvor identifizierten Genres und mitunter Regionen in Auftrag gegeben werden. 
Diese kann zu einer deutlich fundierteren Bestätigung oder Ablehnung der vorherigen Vorabentscheidung beitragen. 
Durch die Erstellung einer groben Marktanalyse und der Visualisierung derer Daten kann die Anzahl bzw. der Umfang detaillierterer, teuerer und zeitintensiverer Marktanalysen minimiert werden.
Entsprechend sollte auch der Fokus einer Visualisierung auf Zeit- und Kosteneffizienz bei gutem Nutzen für zuvor beschriebenes liegen.

%aus Einleitung eingefügt
Wichtig dabei ist das Bedürfnis nach Intuitivität der Visualisierungen, Schnelligkeit im Verständnis und der Erstellung sowie des vergleichweise geringen Preises für sie zu berücksichtigen. 
Die Visualisierungen müssen immer wieder schnell verständlich und erklärbar für andere sein, auch auf den ersten Blick.

Zunächst soll die Frage nach den angebotenen Videospielen der Publisher und ihrer Konkurrenz in einem ersten, qualitativen Überblick beantwortet werden. 
Der Fokus liegt aufgrund der gegebenen Daten und deren Übersichtsmöglichkeiten auf der Identifikation der von einem Publisher angebotenen Genres und der Videospiele, die jenen zugeordnet werden können. 
So lässt sich die vergangene Wahl des Schwerpunktes der Publishers bzgl. des Genres und der in diesem angebotenen Spiele erkennen. 
Besonders für größere Publisher mit vielen Videospielen oder vielfältigen Genres ist ein solcher Überlick sinnvoll.
In Kombination mit einer Übersicht über die konkurrierenden Publisher in derselben Art, kann ermittelt werden, welcher Publisher eine starke Konkurrenz in welchem Genres darstellen kann, welche Genres stark oder weniger stark angeboten werden und wieviele Videospiele ihnen zugeordnet werden können.
Somit können erste potenzielle Marktlücken, aber auch Sättigungen identifiziert werden, die für eine zukünftige Ausrichtung des Publishers relevant sein können. 
Es wird die Annahme getroffen, dass die Zielgruppe häufig und in unregelmäßigen Abständen einen Überblick über ihre Videospiele, deren Genres und ihre Konkurrenz auf den einen Blick benötigt. 
Dies kann sowohl im persönlichen Rahmen des Betrachtens durch Einzelpersonen des oberen und mittleren Managements der Fall sein als auch im Rahmen von Präsentationen der Analyse für das obere Management und wichtige Stakeholder.
Durch die Übersicht soll bei der Zielgruppe ein mentales Modell der Hierarchie und Zugehörigkeit der Videospiele zu Genres und dieser Genres zu den Publishern entstehen. 

Die zweite Unterfrage beschäftigt sich mit der Gewinnung von Erkenntnissen über Muster, Cluster, Ausreißer oder Korrelationen in den Verkaufszahlen zwischen den einzelnen Regionen der Welt und über alle Regionen hinweg.
Dazu müssen die Daten so aufbereitet werden, dass sie in einer oder mehreren Darstellungen durch die Positionierung im Raum miteinander vergleichbar sind. 
Alle Erkenntnisse müssen schnell und einfach ersichtlich sein, sodass im Kopf der Personen der Zielgruppen leicht Bilder entstehen können.  
Datenpunkte in jedweder Form müssen sich zur Bildung von Mustern und Clustern zu Gruppen mittels Farbe, Position oder Form zusammenfinden können, die von anderen unterscheidbar sind. 
Ausreißer können ebenso erkannt werden.
Korrelationen in den Verkaufszahlen zwischen den Regionen werden weiterführend mental meist am besten als Gruppe von Datenpunkten im Raum positioniert durch Absteigen, Aufsteigen oder Verstreuung erkannt.

Es ist schlussfolgernd sinnvoll, der Zielgruppe mittels einer Visualisierung erst eine Übersicht auf alle Regionen, sprich Attribute, auf einmal zu gewähren, in der Muster und Cluster sowie Ausreißer durch Markierung mit Farbe, Form oder Position über alle Regionen hinweg erkennbar sind. 
Mental soll ein Blick von oben auf die Welt entstehen. 
So kann sich ein zusammenhängendes Bild über die Verkaufszahlen in den Regionen ergeben ähnlich zu einem Motiv, dass sich durch das Verbinden von Zahlen in einem Zahlenbild für Kinder ergibt.
Werden verschiedene solcher Bilder übereinandergelegt, hier entsprechend der konkurrierenden Videospiele, lassen sich Unterschiede erkennen.
Sollte es zudem bspw. einen Ausreißer geben, der besonders gute Verkaufszahlen in dieser Region und global aufweist, dann deutet das auf eine Bedarfsdeckung dieser Region durch das entsprechende Videospiel hin. 
Eine weitere Investition in diese Region in dem Genre wäre entsprechend zu überdenken. 
Ist aber bswp. ein Videospiel des Publishers X in dieser Region der Ausreißer und besitzt auch global gute Verkaufszahlen, so ist eine Investition in weitere Videospiele dieser Reihe eine sinnvolle Option.
In einem mentalen Modell können diese Analysen durch räumliche Abbildungen bzw. Abbildungen der Daten auf Positionen in einem Kontext dargestellt werden.

Mittels einer weiteren Visualisierung soll der Blick konkretisiert und auf zwei interessante Regionen fokussiert werden. 
Mental soll in das zuvor entstandene Bild hineingezoomt werden, um detaillierter auf zwei ausgesuchte Regionen des Genres blicken zu können.
Sie können genauer auf Cluster und Ausreißer geprüft und die Art der möglichen Korrelation konkretisiert werden.
Eine weitere Betrachtung der Gründe für letztere und der Ausnutzung dieser Korrelation in fortführenden Studien ist sinnvoll. 
Für die Bearbeitung dieser Unterfrage und vor allem der leichten Erkennung von Korrelationen und Clustern ist ein räumliches mentales Modell der Abbildung der Daten auf Positionen sehr wichtig und notwendig.
Die Bestimmung der zu untesuchenden Regionen für weitere Investitionen in fortführenden Marktstudien kann so besser eingegrenzt werden.

Untergeordnet, jedoch sekundär auch für die Beantwortung der Hauptfrage wichtig, ist die Darstellung und Erkennbarkeit der Performanz der Videospiele des Genres in einer solchen Weise, dass einzelne Spiele konrekt erkannt und zugeordnet werden können.
So können Verkaufszahlen und damit Erfolge einzelner Spiele im Genre konkret verglichen werden. 
ufszahlen seiner Spiele feststellen.  
Bei der Bearbeitung dessen helfen dieselben mentale Modelle sowohl zur konkreten Ansicht als auch der gleichzeitigen Darstellung aller Attribute wie zur Beantwortung des zweiten Unterziels. 
Wichtig dabei ist die Möglichkeit der Identifizierung der Datenpunkte durch Beschriftung.

%Analysieren sie die konkreten Anwendungsaufgaben, die die Lösung des Zielproblems durch die Anwender:innen bearbeitet werden müssen. 
%Welche sinnvollen mentale Modelle helfen den Personen bei der Bearbeitung. 
%Welche Visualisierungen helfen den Personen, die die Software verwenden, sinnvolle mentale Modelle aufzubauen. 
%Sind diese mentalen Modelle für sie notwendig, um die Aufgaben lösen zu können? Gehen sie bei ihrer Argumentation von den Anwendungsaufgaben aus und kommen sie dann zu den mentalen Modellen, deren Aufbau durch Visualisierungen unterstützt wird. 
\subsection{Anforderungen an die Visualisierungen}
Aus den zuvor analysierten Anwendungsaufgaben leiten sich Anforderungen an die Visualisierungen ab. 
Allgemein sollen alle Visualsierungen effizient, expressiv und angemessen sein. %bibliografie 
So ist bedingt durch die Zielgruppe und die Nutzung der Visualisierungen im Rahmen von Präsentationen vorrangig wichtig, dass die Inhalte intuitiv und ohne viele Erklärungen erkennbar werden. 
Weiterhin muss die Aussagekraft eindeutig sein, um unbeabsichtigte Suggestionen zu vermeiden. 
Zuletzt sollte besonders im Kontext einer betriebswirtschaftlich orientierten Analyse auch die Angemessenheit gegeben sein. 
Zwar ist eine hier anvisierte Marktanalyse essentiell für ein Verlagshaus und entsprechend auch bezüglich des Ressourchenverbrauchs wertvoll, jedoch werden besonders hier die Kosten in Bezug zur Effektivität gestellt. 
Es ist für eine grobe Marktanalyse wenig sinnvoll, sehr komplexe, tiefgründige Visualsierungen mit hohem Ressourcenverbrauch zu nutzen, wenn es besonders in Bezug auf die Zielgruppe sinnvollere, eventuell weniger komplexe, 
aber dafür leichter und schneller verständliche Visualisierungstechniken gibt, die entsprechend weniger Ressourcen in der Erstellung benötigen. 
Selbiges gilt für den Detaillierungsgrad der Daten, die für die Anwendungsaufgaben wie zuvor definiert grober sein dürfen und erst nach Entscheidung mithilfe der hier präsentierten Visualisierungen detaillierter und durch weitere Visualisierungstechniken komplexer dargestellt werden sollen.

Zur Beantwortung der einleitend ersten gestellten und zuvor analysierten Unterfrage ist eine überlicksartige, hierarchische Darstellung der Daten zu Publishern, den von ihnen bedienten Genres und in diesen angebotenen Videospielen notwendig.  
Die Darstellung sollte von den jeweiligen Publishern ausgehend spezifischer werden. 
Zur schnellen Erkennung der Videospielangebote sollte nicht nur die Anzahl und Art der Genres je Publisher berücksichtig werden. 
Aufgrund der mentalen Gewichtung und Bewertung der abgedeckten Genres muss auch die Darstellung der Videospieltitel je Genre des jeweiligen Publisher vorgenommen werden. 
Zur Ermittlung der Konkurrenz ist es nötig, alle Publisher in der Übersicht in derselben Art und Weise abzubilden. 
Neben diesen Anforderungen ist jedoch die schnelle Übersicht, vor allem über das eigene Verlagshaus die wichtigste. 
Es muss ohne viel Nachdenken und viele Erklärungen sowohl für die primäre Zielgruppe, als auch für die sekundäre, also Stakeholder des Verlages, ersichtlich werden, welche Genres der Publisher anbietet, welche und wieviele Spiele unter diesen Genres katgorisiert sind und wie die Konkurrenz aufgestellt ist.
Die Hauptaufgabe der ersten Visualisierung liegt in der Präsentation der schon klar bestimmten Daten des Sachverhaltes der Publisher, der von ihnen abgedeckten Genres und den ihnen zugeordneten Videospielen in einer schnell verständlichen, hierarchischen Darstellung.

Aus der Analyse der zweiten Unterfrage ergibt sich die übergeordnete Anforderung der visuellen Analyse der Datenmenge für die zweite Visualisierung. 
Es muss in der Visualisierung in einem Vergleich auf einen Blick erkennbar werden, wie sich welche Spiele eines Genres in welchen Regionen der Welt verkauften. 
Somit ist es von Nöten, alle in den Daten gegebenen Regionen in einer Darstellung abzutragen, in der sie miteinander verglichen werden können. 
Die Visualisierung mehrdimensionaler Daten ist essentiell.
Für eine bessere Vergleichbarkeit der Daten soll nach Genre kategorierisiert werden und dieses je nach Betrachtungswunsch ausgewählt werden.
Um die geforderten Muster und Cluster über alle Regionen hinweg zu visualisieren, ist eine visuelle Zuordnung zueinander mittels Farbe und Position wichtig. 
Muster sind bspw. gut durch Überlappungen der Datenpunkte erkennbar.
Durch die exponierte Positionen können zudem Ausreißer erkannt werden. 
Es braucht eine deutliche Unterscheidung der Spiele untereinander, bspw. mittels Farbe, sowie eine Beschriftung mit den Titeln der Videospiele, um die untergeordnete genaue Erkennbarkeit aller Videospiele zu ermöglichen.
Die Ausprägung der Verkaufszahlen muss zumindest grob ablesbar sein, wobei der visuelle Vergleich zur Unterstützung des mentalen Modells des übersichtsartigen Vergleichens vor expliziten, genauen Zahlen Vorrang hat.

Die übergeordnete Anforderung an die dritte Visualisierung ist wieder die visuelle Analyse der Datenmenge. 
Hierbei soll es der Zielgruppe möglich sein, im Detail Korrelationen zwischen den Regionen der Videospielverkäufe zu erkennen. 
So kann erkannt werden, wie die Zusammenhänge zwischen bspw. sehr guten Verkaufszahlen einer Region zu den Verkaufszahlen einer anderen gegeben sind. 
Weiterhin sollen noch einmal präziser die Verkaufszahlen der Videospiele der Publisher in einem Genre im Kontext dargestellt werden. 
Auch hier besteht die Anforderung, die Analyse und Visualisierung im Kontext des Genres vorzunehmen, damit die Zielgruppen die einzlenen Videospiele konkreter vergleichen können.
In dieser Visualisierung muss die Relation von zwei Regionen zueinander dargestellt werden, sowie der Titel des entsprechendes Videospiel ersichtlich werden. 
Zum präzisen Vergleich von Verkaufszahlen in den Regionen müssen auch die exakten Zahlen zu finden sein und möglichst nicht nur rein visuell abgetragen werden. 
Für eine Visualisierung der möglichen Cluster und Zusammenhänge zwischen den Verkaufszahlen konkurrierender Spiele, müssen alle Titel aller Publisher des Genres vertreten sein.
Durch die gewünschte detaillierte Analyse ist eine intuitiv verständliche Visualisierung sowie ein gut verständliches Design dieser wichtig. 

Über alle Visualisierungen hinweg ist eine einheitliche Wahl der Farbgebung zur Verbesserung der Effektivität und Effienz nötig. 
Um Probleme bei Sehschwächen der diversen Zielgruppe zu vermeiden, ist die Wahl nur einer Farbe zur Hervorhebung und Darstellung der Datenpunkte sinnvoll. 
Durch Abstufung mittels Deckkraft sind sie auch für Farbblinde erkennbar und verdeutlichen zudem bestimmte Erkenntnisse.
Sogenannte \textit{Just Noticable Differences}, also gerade so erkennbare Farb- und Kontrastunterschiede, in der Gestaltung der Website an sich helfen bei der Führung des Auges des Betrachters.

Allen spezifischen Anforderungen voran steht die Anforderung an die Angemessenheit. 
Die eingesetzten Visualisierungstechniken sollten günstig sein und deshalb einfacher gestaltet. 
Gleichzeitig sollten sie durch die Notwendigkeit von schnellen Entscheidungen zügig erstellbar und ebenso schnell und einfach verständlich sein, damit auch die in den Visual Analytics ungeübten Personen der Zielgruppe ein möglichst breites und präzises Verständnis für die Marktsituation, Verkaufszahlen und Korrelationen zwischen den Regionen erhalten können.

%Leiten sie Anforderungen an das Design der Visualisierungen ab, die sich durch ihre Analyse des Zielproblems ergeben.
\subsection{Präsentation der Visualisierungen}
Zur Erfüllung der Anaylsen und Anforderungen werden die Visualisierungstechniken \textit{explizites Baumdiagramm}, \textit{parallele Koordinate} und \textit{Scatterplot} gewählt.
Nachfolgend werden die ausgesuchten Visualisierungen vorgestellt und diskutiert. 
%Präsentieren sie die visuelle Abbildungen und Kodierungen der Daten und Interaktionsmöglichkeiten. 
%Sie müssen  begründen, warum und wie gut ihre Designentscheidungen die erstellten Anforderungen erfüllen. 
%Weiterhin müssen sie begründen, warum die gewählte visuelle Kodierung der Daten für das zulösenden Problem passend ist.
%Typische Argumente würden hier auf Wahrnehmungsprinzipien und Theorie über Informationsvisualisierung verweisen. 
%Die besten Begründungen diskutieren explizit die konkrete Auswahl der Visualisierungen im Kontext von mehreren verschiedenen Alternativen. 
%Machen sie hier nicht den Fehler, einfach nur Visualisierung aus den vorgegebenen Bereichen zu diskutieren, weil das in der Regel nicht sinnvoll ist.
%Wenn sie sich für einen Scatterplot entschieden haben, ist ein Zeitreihendiagramm in der Regel keine Alternative.
%Diskutieren sie also nicht einfach Zeitreihendiagramme, weil sie in den Anforderungenen an das Projekt neben Scatterplots stehen, sondern suchen sie nach echten alternativen Visualisierungen, die zum Aufbau eines vergleichbaren mentalen Modells führen. 
%Diskutieren sie die Expressivität und die Effektivität der einzelnen Visualisierungen. 
%Die eben beschriebenen Präsentationen und Begründungen sollen für jede der drei folgenden Visualisierungen durchgeführt werden. 

\subsubsection{Visualisierung Eins}
Zur Erfüllung der Anforderungen wird für die erste Visualisierung ein explizites Baumdiagramm gewählt.
Mit einem solchen können Hierarchien gut, übersichtlich und auf allgemein bekannte Weise dargestellt werden. 

%warum Baum + Theorie + Wahrnehmungsprinzip

In der Visualisierung gibt es einen Wurzelknoten, der als Kindknoten die Publisher hat, die wiederum als Kindknoten die vom jeweiligen Publisher abgedeckten Genres besitzen. 
Zur besseren Übersicht und Vergleichbarkeit werden nur diejenigen Genres pro Publisher abgebildet, die dieser Publisher anbietet und in alphabetischer Reihenfolge aufgeführt. 
Wichtig dabei ist, dass keine sich überkreuzenden Pfade entstehen respektive es keinen Kindknoten gibt, der mehrere Elternknoten hat. 
Dies dient der Erfüllung der Anforderungen und der Übersichtlichkeit der Visualisierung.
Die Genreknoten besitzen wiederum Kindknoten, die die Titel der Videospiele, die dem entsprechenden Genre zugeordnet sind, tragen. 
So entsteht die gewünschte hierarchische Übersicht über das Angebot jedes Publishers, welche durch die Beschriftung und Hervorhebung bei Hovern über jeden Knoten unterstützt wird.
Die Anforderungen an eine schnelle, intuitive und für jeden verständliche Übersicht werden größtenteils erfüllt. 
Jedoch besteht der Nachteil der Größe des Baumes durch die vielen Publisher- und Genreknoten. 
Trotzdem kann der Baum durch seine Nutzung der Hierarchie, also Abbbildung der nominalen Daten auf eine Position im Raum, sowie durch die Nutzung von Verbindungslinien zur Unterstützung der Positionierung und damit Erstellung des mentalen Modells der gewünschten Hierarchie die Anforderungen an Effektivität gut erfüllen.
Entsprechende visuelle Variablen werden als sehr effektive Abbildungsmethodik für nominale Daten eingestuft. 
Expressiv und damit ohne viele Erklärungen und Missverständnisse ist diese Technik zudem.
Jedoch leidet die Expressivität etwas unter der Menge an Knoten und damit der Größe des Baumes.
Auch angemessen ist eine Visualsierung als explizites Baumdiagramm, da es wenig Ressourcen fordert. 

Alternativen zu dieser Darstellung sind implizite Baumdiagramme, radiale Baumdarstellungen und hyperbolische Bäume. 
Implizite Baumdarstellungen können ebenso wie explizite die Hierarchien zwischen Eltern- und Kind-Knoten darstellen. 
%Grund
%umschließen, unüberischtlichkeit, nicht so direkt verständlich mit vielen Erklärungen & beschriftungen
Radiale und hyperbolische Bäume sind gute Alternativen für Darstellungen mit vielen Knoten wie es in dieser Visualisierung auch der Fall ist. 
%Grund
%macht Sinn wegen anzahl knoten, aber probleme mit überschneidungen, schnelles lesen, platz knoten wächst linear mit umfang aber Knoten können exponentiell wachsen
%Platzierungsprobleme nicht, aber durch Darstellung evtl schwieriger zu veregleich mit Konkurrenz, angemessenheit nicht 100, da viel aufwand ressourcen zum erstellen und mehr zum lesen bei der eig überblicksartigsten & einfachsten Visualisierung
Vor allem letztere Alternativen bieten in Bezug auf Expressivität, Effektivität und Angemessenheit nahezu dieselben Vorteile wie die expliziten Bäume. 
Letztlich wird die Entscheidung für ein explizites Baumdiagramm trotz des Nachteils der vielen Knoten, die durch hyperbolische und radiale Bäume übersichtlicher gestaltet werden könnten, aufgrund der Übersichtlichkeit durch Intuitivität der expliziten Baumdarstellungen getroffen. 
Dies mag paradox wirken, jedoch sind explizite Bäume durch ihr häufigeres Vorkommen in allen Lebenssituationen noch ein wenig intuitiver und damit effektiver verständlich als die Alternativen. 
Da die Anforderung vor allem auf schnellem Erkennen und Wiedererkennen insbesondere des eigenen Verlagshauses liegt, wird diesem Aspekt eine hohe Gewichtung gegeben.

\subsubsection{Visualisierung Zwei}
Als zweite Visualisierungstechnik zur Teillösung der zweiten Unteranwendung werden die parallelen Koordinaten gewählt.
%warum parallele + Tehorie + Wahrnehnumgsprinzip

Für diese Visualisierung werden die Datenpunkte als Abfolge von Liniensegmenten dargestellt, wobei der Attributwert der jeweilige Endpunkt des Liniensegments ist. 
Visualisiert werden die Wertebereiche der Attribute \textit{North America}, \textit{Europe}, \textit{Japan}, \textit{Rest of World} und \textit{Global}, welche die Verkaufszahlen in Millionen Einheiten der Videospiele in den verschiedenen Regionen darstellen.
Durch die Verbindung der Liniensegmente und den jeweiligen Endpunkten auf den parallelen Liniensegmenten bzw. Achsen kann die Zielgruppe die Attributwerte dieser fünf Attribute ablesen. 
Durch die Darstellung aller Videospiele als je eine solche Sequenz von Liniensegmenten können Unterschiede erkannt werden. 

Wie in den Anforderungen beschrieben, soll es zur besseren Vergleichbarkeit der Datenpunkte und damit Videospiele eine Filterung nach Genres geben, die mittels eines Drop-Down-Menüs umgesetzt wird. 
Da es bei parallelen Koordinaten zu Problemen mit dem Überzeichenen von Liniensegmenten kommen kann, wenn es zu viele Datenpunkte gibt, schafft die Filterung hierbei zudem Abhilfe. 

Durch die Implementierung einer Art umgekehrten Röntgenstrahleneffekt durch die Nutzung einer Farbe mit wenig Deckkraft kann dieses Problem zusätzlich gelöst werden. 
Zudem können Überlappungen dadurch besser zur Visualisierung der geforderten Muster, aber vor allem Cluster genutzt werden. 
Je dunkler bestimmte Segmente, desto mehr Datenpunkte laufen hier überein. 
Auch Überkreuzungen von Datenpunkten sind einfach durch die dunklere Farbe am Kreuzungspunkt erkennbar. 

Zur Erkennung des Videospieltitels und des Publishers, wird beim Hovern über ein Liniensegment beides eingeblendet. 
Die Verkaufszahlen sind anhand der Achsenbeschriftung ungefähr ablesbar, beim Hovern allerdings konkret. 
So überwiegt wie gewünscht das visuelle Erkennen und Vergleichen, aber auch die genaue Erkennbarkeit der Videospiele ist gegeben.
Zudem wird der entsprechende Datenpunkt mittels Farbe hervorgehoben, um ihn eindeutig erkennbar zu machen. 
Dadurch können die konkreten Anforderungen sowie die allgemeinen Anforderungen an Expressivität und Effektivität an die zweite Visualisierung  erfüllt werden. 

Quantitative Daten werden weiterhin am besten mit den visuellen Variablen Position, aber auch Orientierung dargestellt.
Beides wird in den parallelen Koordinaten umgesetzt, die Orientierung vor allem durch die Sequenz an verbundenen Liniensegmenten, die parallelen Achsen und den entstehenden Kontext der Daten. 

Die Expressivität wird zusätzlich durch die eindeutige Beschriftung der Liniensegmente beim Hovern, die Einfärbung und die eindeutige Beschriftung der Achsen erreicht. 
Auch die weiteren Anforderungen vor allem an die Vergleichbarkeit und Erkennung von Mustern in einem Genre über alle Regionen hinweg in einer leicht verständlichen Art und Weise wird erfüllt.
%Zusätzlich können leicht Cluster erkannt werden, wenn bestimmte Videospiele, bspw. eine Reihe in einem Genre, alle in allen Attributen, sprich Regionen, gute Werte aufzeigen. 
%Auch dies bietet Aufschlüsse über den Markt und entsprechende Konkurrenz.

Eine der Alternativen zu sind parallelen Koordinaten die Icon Techniken. 
Sie werden jedoch als weniger gut im Vergleich zu den parallelen Koordinaten eingeschätzt.
%Grund
Zuletzt bieten Sternkoordinaten und auch Polar Plots alternative Möglichkeiten der Darstellung mehrdimensionaler Daten. Allerdings
%Grund S. 232

\subsubsection{Visualisierung Drei}
Zuletzt wird zur weiteren Teillösung der zweiten Unteranwendung die Visualisierungtechnik Scatterplot umgesetzt. 
%Scatterplot + Grund + Theorie + Wahrnehmungsprinzip

In diesem Scatterplot wird auf die X-Achse je nach Auswahlwunsch des Betrachters je der Wertebereich der Attribute \textit{North America}, \textit{Europe}, \textit{Japan}, \textit{Rest of World} und \textit{Global} abgebildet.
Die Abbildung der Wertebereiche auf die Y-Achse funktioniert auf dieselbe Art. 
So kann der Anwender frei zwischen zwei Attributen seiner Wahl wählen, die er untersuchen möchte.
Es werden je die Datenpunkte visualisiert, die dem per Drop-Down-Menü ausgewählten Genre entsprechen. 

Die Datenpunkte werden klassisch als ungefüllte Kreise dargestellt, wobei sich ihre Position aus den Werten der oben beschriebenen, momentan ausgewählten Atribute bestimmt. 
Beim Hovern über die Punkte werden diese eingefärbt und ein weiteres Attribut, der Titel des Videospiels angezeigt. 
Zur exakten Erkennung der Attributwerte der Verkaufszahlen werden diese hinter dem Titel in Klammern aufgelistet. 
Es wird zuerst der Wert für die X-Achse, dann der Wert für die Y-Achse dargestellt.

Durch diese Designentscheidungen werden die zuvor formulierten Anforderungen erfüllt. 
Die Kodierung der Daten auf die Punkte im Scatterplot, also Koordinatensystem ermöglichen ein Wahrnehmen und damit eine Orientierung im Raum, der mental entsteht und das Verständnis für die Relation der Datenpunkte zueinander verbessert.
Weiterhin ist dadurch die visuelle Darstellung von möglichen Korrelationen zwischen zwei Attributen erkennbar. 

Durch die zweidimensionale Darstellung der Attribute kann eine genauere Betrachtung dieser erfolgen. 
Entsprechend können Korrelationen besser analysiert werden.
Auch die zuvor schon erkannten Muster und Ausreißer können überprüft werden.
Durch die Einblendung der exakten Attributwerte ist auch die untergeordnete Anforderung der genauen Erkennbarkeit der Videospiele erfüllt. 

Die eher klassische Darstellung als Punkte erleichtert das Erkennen von Punktwolken respektive Cluster durch Intuition, insbesondere sollte es Korrelationen zwischen einer Region und den globalen Verkaufszahlen geben. 
Ähnlich wie in Visualisierung zwei werden auch in Scatterplots die für quantitative Daten gut geeigneten visuellen Variablen Position, Orientierung und Gebiet genutzt. %bibliografie

Weiterhin ist diese Darstellungsweise allgemein bekannt, sodass eine schnelle Analyse einzelner Wertpaare, aber auch Korrelationen und Ausreißer möglich ist. 
Dadurch und durch die oben genannten Aspekte ist die Effektivität gegeben. 
Auch expressiv ist diese Visualisierung, da die Darstellung der Datenpunkte als Kreise auch ohne Legende schnell erkannt wird. 
Durch die Einblendung der exakten Werte beim Hovern wird eine mögliche Verwirrung durch Überlagerung verschiedener Symbole, Farben oder Größen der Punkte verhindert. 
Die Wahl einer Farbe zur Darstellung der Datenpunkte, Füllung dieser beim Hovern und Wahl einer nicht einhundertpozentigen Deckkraft zur Ermöglichung der vereinfachten Erkennung der Punkte bei Überdeckung verbessert die Expressivität weiter. 
Durch die unkomplizierte und schnell erstellbare Darstellung ist auch die Angemessenheit gegeben. 

Alternativ bestehen in diesem Kontext keine sinnvollen Visualisierungsmöglichkeiten.
Q- und QQ-Plots stellen Attributwerte nur F-Werten gegenüber bzw. auch gegeneinander, dann jedoch um Verschiebungen zueinander zu erkennen. 
Korrelationstabellen lassen die hier vor allem gewünschten Korrelationen genau analysieren, stellen diese jedoch nicht visuell dar.
So leiden Expressivität und Effektivität stark und die Anforderungen an Intuitivität, sowie Überprüfung von Mustern und Ausreißern sind nicht gegeben. 
und Korrelationstabellen. 

\subsection{Interaktion}
In der ersten Visualisierung werden Knoten bei Interaktion durch Hovern über sie mittels Farbe hervorgehoben sowie beschriftet. 
So ist eine allgemeine Übersicht ohne zu viele Details auf einmal möglich.
Die detaillierten Visualisierungen sind durch die Interaktion des Klicks auf die Weiterleitung erreichbar.
So wird die Struktur der zuerst angebotenen Übersicht und der dann folgenden Detaillierung mittels Verkaufszahlen ermöglicht. 
In dem Parallelen Koordinaten Plot sind die Regionen pro Achse frei wählbar. 
Ein einfacher Achsentausch ist mit steigender Anzahl der Achsen unüberischtlich und beansprucht in der Auswahl zu viel Zeit.
Die freie Wahl ermöglicht der Zielgruppe mehr Schnelligkeit und Einfachkeit in der Wahl der gewünschten Reihenfolge. 
Weiterhin werden die Datenpunkte bei Interaktion durch Hovern über sie mittels Farbe hervorgehoben und die Details außerhalb des Koordinatensystems angezeigt.
Im Scatterplot wird ist die Belegung der X- und Y-Achse ebenso frei mit den vorhandenen Regionen belegbar.
Die Zielgruppe kann somit individuell wählen, welche Regionen durch zuvor betrachtete Parallele Koordianten interessant sind und dargestellt werden. 
Auch hier werden bei Interaktion durch Hovern die Datenpunkte durch Farbe hevorgehoben sowie Details zum jeweiligen Videospiel angezeigt. 
Somit entsteht eine Einheitlichkeit in allen Visualisierungen, vor allem aber in der zweiten und dritten. 

Weiterhin kann durch die Interaktion mit einem Dropdown-Menü zwischen allen drei Visualsierungen gewechselt werden. 
Änderungen der Achseneinstellungen im Parallelen Koordinaten Plot und im Scatterplot bleiben beim Wechsel zwischen diesen bestehen, sodass eine erneute Einstellung nicht nötig ist. 
So sind schnelle Vergleiche und Wechsel zwischen beiden detaillierten Visualisierung möglich. 
Das Baumdiagramm als Ausgangsübersicht ist zudem mittels Interkation mit Link über jede der beiden anderen Visualisierungen möglich.
In den Anforderungen für Visualisierung zwei (Parallele Koordianten) und drei (Scatterplot) ist eine Filterung nach Genre formuliert. 
Diese ist per Interaktion mit Dropdown-Menü und Auswahl eines Genres erreichbar. 
Wird dies in einer der beiden Visualisierungen ausgewählt, wird es in der anderen übernommen ohne bei Wechsel zu dieser Visualisierung nochmal ausgewählt werden zu müssen.
Eine Änderung des Genres in einer der beiden detaillierten Visualisierungen führt zu einer Änderung in der anderen.

\section{Implementierung}
Im folgenden Kapitel wird die Implementierung der zuvor beschriebenen Visualisierungen und Interaktionen beschrieben. 
%Der Code wird durch Kommentare ergänzt, um eine Übersichtlichkeit und Erklärung zu einzelnen Funktionen zu ermöglichen.
Der Code wird in fünf Elm-Module unterteilt, bestehend aus je einem Modul für die drei einzelnen Visualisierungen, einem für Datenstrukturen und einem zur Zusammensetzung und Ausführung des Hauptprogramms. 
\textit{TreeHierarchy} dient als Ausgangspunkt der Anwendung, wobei \textit{MainScatterParallel} den Hauptteil der Anwendung durch die eigentliche Verbindung der Visualisierungen enthält.
Innerhalb der Visualisierungs-Module werden zuerst kleinere Funktionen definiert, die im späteren Verlauf genutzt werden bis allgemeine Definitionen der Plots schließen. 
Über alle Module hinweg werden diverse Bibliotheken genutzt, die zu Beginn jedes Moduls importiert werden.  
%Die Bibliotheken \textit{elm/core} und \textit{elm/browser} werden nur für MainScatterParallel importiert. 
%\textit{elm/http} wird von MainScatterParallel, Data und TreeHierarchy verwendet, \textit{elm/html} von MainScatterParallel, TreeHierarchy und Scatterplot.
%Zur Decodierung von JSON-Dateien für TreeHierarchy wird die Bibliothek \textit{elm/json} in Data genutzt, für die Decodierung von CSV-Dateien für Scatterplot und ParallelPlot in Data \textit{lovasoa/elm-csv} und \textit{ericgj/elm-csv-decode}.
%Die Bibliothek \textit{elm-community/typed-svg} wird für das Zeichnen der Plots in Scatterplot, ParallelPlot und TreeHierarchy gebraucht ebenso wie die Bibliothek \textit{gampleman/elm-visualization}. 
%\textit{avh4/elm-color} wird nur in ParallelPlot und TreeHierarchy genutzt. 
%\textit{elm-community/list-extra} und \textit{folkertdev/one-true-path-experiment} sind nur für das Zeichnen des ParallelPlots wichtig.
%Zur Berechnung des Layouts von Bäumen wird \textit{wasdacraic/elm-tree-layout} in TreeHierarchy genutzt und \textit{zwilias/elm-rosetree} zum Speichern der Bäume in einer Datenstruktur in TreeHierarchy und Data importiert.
           
Insgesamt zeigte sich die Implementierung der Anwendung als zeitaufwändig, da trotz der umfangreichen Übungsaufgaben die Routine und teils das Verständnis im Programmieren mit Elm fehlte.
Die Übungsaufgaben waren jedoch enorm hilfreich und dienten oft als Grundlage der Funktionen.
Im gesamten Projekt gab es neue Erkenntnisse und Verständnis für die Funktionen der Sprache und des Quellodes, sodass sich ein zusätzlicher großer Lerneffekt einstellte.
Trotz der guten Vorlagen der Abgaben der Autorin zu den Übungsserien traten einige Schwierigkeiten auf, die in den folgenden Unterkapiteln Erwähnung finden.
          
\subsection{Data.elm}
Dieses Modul dient der Deklaration nahezu aller verwendeten Datentypen sowie der Implementierung der Dekodierungen. 
Einzig \textit{type Model} und \textit{type Msg} sind aus Gründen der Übersichtlichkeit und Verständlichkeit des Codes in \textit{MainScatterParallel} dort nicht vertreten. 

Zunächst sind die Funktionen \textit{decodeGameSales}, \textit{csvString\_to\_data} und \textit{gamesSalesList} zur Decodierung der CSV-Dateien implementiert. 
Die einzelnen Felder der CSV-Datei werden als String decodiert und in eine Liste vom Typ GameSales geschrieben.
Darauf folgt die Decodierung der JSON-Datei mittels \textit{treeDecoder}.
Beide Decoder finden im \textit{update} in \textit{MainScatterParallel} zur initialen Überschreibung der Daten ihre Anwendung.
Sie basieren den für die Übungen sieben und neun verwendeten Decodern und wurden an die hier verwendeten Daten angepasst.

Es folgt die Deklaration jener Datentypen, die für alle Visualisierungen außer die \textit{TreeHierarchy} wichtig sind. 
\textit{Type alias GameSales} definiert alle Daten der CSV-Datei wie sie durch den Decoder dekodiert werden und später im Model in \textit{MainScatterParallel} weiterverwendet werden. 
Die Deklarationen \textit{type RegionType} und \textit{type PlotType} als CustomType werden für die Auswahl der Achsen in den Visualisierung zwei und drei sowie die Auswahl der anzuzeigenden Visualisierung benötigt. 
Sie werden im \textit{type Model} und im \textit{type Msg} in \textit{MainScatterParallel}, im \textit{regionFilter} in \textit{Scatterplot} und im \textit{multiDimenData} in \textit{ParallelPlot} verwendet. 
Es folgen die durch die Übungsserien bekannten Datentypen für die spezifischen Visualisierungen. 
%Somit sind alle drei Visualisierungen modular zusammengesetzt.

Zuletzt werden die für die Buttons wichtigen Funktionen zur Typkonversion von einem String zum jeweiligen CustomType und, mit Ausnahme vom PlotType, umgekehrt beschrieben. 
Dies ist nicht nur zur Erstellung von Buttons, die nur Strings erlauben, nötig, sondern auch zur bspw. korrekten Anzeige von aktuell ausgewählten Regionen in der Folge.
Weiterhin ist eine Konversion von einem RegionType zu einem Tupel aus GameSales und Float für die Funktion \textit{multiDimenData} in ParallelPlot notwendig.

\subsection{TreeHierarchy.elm}
Für das Modul können große Teile des von der Autorin in Übungsserie neun erstellten Quellcodes verwendet werden und müssen lediglich minimal auf die verwendeten Daten angepasst werden. 
Begonnen wird mit den grundlegenden Funktionen \textit{main}, \textit{init} und \textit{update}. 
Die Anwendung wird initialisiert, die Daten unter Anwendung des Decoders für JSON-Dateien geladen und die jeweils auszuführenden Aktionen bei korrektem oder inkorrektem Laden der Daten bestimmt. 
Es folgt eine Helferfunktionen \textit{convert} zur Konvertierung des Baumes basierend auf den Vorgaben zu Übung neun.

\textit{Line} und \textit{point} definieren die Linie zum Zeichnen der Verbindungen von Eltern zu Kindknoten bzw. zur Darstellung der Knoten an sich sowie die um 75° rotierte, fett gedruckte Beschriftung. 

\textit{TreePlot2} zur Zeichnung des Baumes ohne Scrollen und \textit{treePlot} zur näheren Ansicht des Baumes werden folgend implementiert. 
Sie unterscheiden sich einzig in der prozentualen Vergrößerung der global definierten Höhen und Breiten um 200 bzw. 150 Prozent im \textit{treePlot}.
Auch die Plots an sich basierend stark auf dem Quellcode der Autorin zu Übung neun. 
Mittels lokaler Funktionen wird das Layout des Baumes bestimmt, bevor die Abhängigkeiten des Baumes mit Hilfe von \textit{Dict.get} zum Erhalt der X- und Y-Werte berechnet werden, um die Pfade zwischen Eltern- und Kindknoten zu zeichnen. 
Die Funktionen \textit{checkRootNegative} und \textit{nodeValuesPath} verhindern das fehlerhafte Zeichnen einer Linie zum nicht existenten Elternknoten der Wurzel.
Nachfolgend werden die berechneten Verbindungen mittels Anwendung zuvor beschriebener Funktion \textit{line} auf \textit{nodeValuesPath} gezeichnet.
Berechnete Knoten werden durch Anwendung von \textit{point} auf \textit{nodeValues} gezeichnet. 
Die Definition des Designs inklusive Hovern wird im global definierten CSS \textit{cssTree} vorgenommen. 

In der \textit{view}-Funktion werden lokal die Daten in einen Baum konvertiert, bevor dies neben dem CSS den beiden Zeichnungsfunktionen des Baumes übergeben und diese dann ausgegeben werden können.
Die Anordnung sowie der Designstil der HTML-Elemente ist an \textit{MainScatterParallel} angepasst. 
%An den entsprechenden Stellen in der Abfolge der gewünschten Texte werden die beiden Baumhierarchien gezeichnet, wobei \textit{treePlot} einen Balken zum horizontalen Scrollen erhält.

Zuletzt werden global allgemeine konstante Einstellungen zum Plot definiert. 
Diese Definitionen basieren mit Unterschied bestimmter Werte auf den Vorgaben in der Vorlage zum Arbeiten mit Bäumen aus Übung neun und den Abgaben der Autorin zu Übung neun. 

Die Implementierung der expliziten Baumhierarchie bereitete keine Schwierigkeiten, da sehr viel auf den abgegebenen Codes der Autorin zu Übung neun basiert. 
Lediglich der Decoder in \textit{Data} musste an die Datenfelder angepasst, der Link zum Laden der Daten geändert sowie die besten Design- und Darstellungsweisen, inklusive der zweifachen Darstellung des Baumes, gefunden werden. 
Das CSS wurde gegensätzlich global definiert und das allgemeine Design Seite im \textit{view} an jenes aus \textit{MainScatterParallel} angepasst. 

\subsection{ParallelPlot.elm}
Dieses Modul beginnt nach dem Import mit drei Funktionen zur Zuordnung der Daten zum benötigten Datentyp und der Reduzierung der Datensätze um solche, die Null-Values enthalten. 
Letzteres ist nach der manuellen Vorverarbeitung der Daten zwar nicht zwingend notwendig, jedoch kann so auf Fehler und Änderungen der Rohdaten eingegangen werden. 

Die erste Hilfsfunktion dient mittels \textit{Maybe.map2} dem Pipen.
Bsierend auf einer normalen \textit{Maybe.map5} Funktion wird die zweite Hilfsfunktion um einen Parameter erweitert. 
Beide begründen sich aus den Dokumentationen der Elm-Core-Bibliothek zu Maybe.%\cite{EvanCzaplicki.o.J.} 
Da es kein \textit{Maybe.map6} gibt, wird in \textit{helpMapBig} die vorherige Hilfsfunktion angewandt, um dieses Problem zu umgehen. 
%So kann transformiert werden wie es eine Maybe.map6 Funktion machen würde, wenn es sie gäbe.
\textit{AssignmentAndReduce} wiederum wandelt schließlich lokal mithilfe der \textit{helpMapBig} die Daten vom Typ GameSales zu Maybe GameSales um und reduziert diese um die Datensätze mit fehlenden Werten mittels \textit{List.filterMap}.
Diese Funktion basiert auf den Abgaben der Autorin zu Übung sechs.

Die Funktion \textit{multiDimenData} dient zusammen mit der lokalen Funktion \textit{multiDimFunction} aus \textit{MainScatterParallel} der freien Auswahl und Belegung der Achsen sowie der Anwendung der Filterung nach Genre. 
Das Grundgerüst beider Funktionen basiert auf den Abgaben der Autorin zu Übung sechs. 
Der dort implementierte Achsentausch wurde aufgrund der Anforderungen in diesem Projekt geändert.
\textit{MultiDimenData} benötigt als Eingabe eine Liste von GameSales sowie fünf RegionTypes als die auszuwählenden Regionen für die Achsen. 
Weiterhin braucht es den Namen des Spieles und den Publisher des Spieles zur Anzeige dieser im späteren Plot. 
Der Typ begründet sich aus dem \textit{type alias GameSales}.
Zusätzlich sind fünf Namen, also Beschriftungen als String nötig. 
Begründet ist dies in der Definition der Recordfelder in \textit{type Model} sowie dem für das \textit{update} benötigten Datentypen \textit{(RegionType, String)}.
Die Funktion muss den Typ \textit{MultiDimData} zur Übergabe an \textit{scatterplotParallel} und Zeichnung des Plots ausgeben. 
Eine besondere Schwierigkeit bei der Konstruktion der Funktion lag im Umfang dieser und im Verständis des benötigten Datentyps. 
So muss der RegionType innerhalb der Funktion mittels \textit{regionTypeToAxisAnnotation} nochmals umgeschrieben werden. 
Die Funktion wird in \textit{MainScatterParallel} in der lokalen Funktion \textit{multiDimFunction} angewandt und bekommt dort die von ihr benötigten Daten übergeben.

Es folgt der \textit{scatterplotParallel}, der nahezu unverändert aus den Abgaben der Autorin zu Übung sechs übernommen werden kann. 
Es werden unter anderem mittels lokaler Funktionen die Achsen in X-Richtung positioniert sowie folgend inklusive Beschriftungen in einem umgebenden Rechteckt gezeichnet.
Die Zeichnung der Datenpunkte wird durch \textit{Shape.line} und \textit{Shape.linaerCurve} realisiert.
Eine Änderung ist der beim Hovern über einen Datenpunkt anzuzeigende Text. Er wird lokal beschrieben bevor dieser Funktion zuletzt die entsprechenden Daten übergeben werden.

Auch hier wird ein globales \textit{cssParallel} definiert.
Zur Enstehung des gewünschten Röntgeneffektes zur erleichterten Identifikation der Cluster und Auffälligkeiten der Datenpunkte wird die Opacity der Datenpunkte auf 0.5 festgelegt.
Durch das Einfärben in einem stärkeren Grünton beim Hovern und dem Einblenden der Details zum Datenpunkt lassen sich die einzelnen Videospiele besser nachvollziehen.

Die Implementierung schließt mit den generellen Einstellungen für den Plot, welche mit Ausnahme bestimmter Werte auf den Abgaben der Autorin zu Übung sechs basieren. 

Die Erstellung des ParallelPlot stützte sich in weiten Bereichen auf den Abgaben zu Übung sechs. 
So bereitete das Zeichnen des Plots wenig Schwierigkeiten. 
Etwas komplizierter war zunächst die Implementierung des Mappings und der Filterung nach Null-Werten durch das fehlende \textit{List.map6}. 
Schwieriger gestaltete sich jedoch die Umsetzung der freien Auswahl der Achsen wie zuvor beschrieben. 
Hier war viel Ausprobieren und Verständnis, welcher Datentyp für welche weitere Funktion benötigt wird und welche durch die vorherige Definition der Datentypen vorhanden waren, nötig.

\subsection{Scatterplot.elm}
Das Modul enthält ähnliche drei Funktionen mit demselben Zweck wie \textit{ParallelPlot}.
Die ersten beiden Funktionen gleichen sich.
\textit{FilterAndReduceGames} basiert jedoch auf den Abgaben der Autorin zu Übung eins.
Die Daten werden zunächst mithilfe von \textit{helpMapBig} vom Typ GameSales dem Typ Maybe Point zugeordnet. 
Die zugeordneten und mit \textit{List.filterMap} gefilterten Daten in \textit{filter} werden dann unter anderem in XyData geschrieben, dass als Output der Funktion \textit{FilterAndReduceGames} definiert ist.
XyData werden benötigt für das Zeichnen des Scatterplots, da sie unter anderem eine Liste Daten des Typs \textit{Point} enthalten.

Die Funktion \textit{regionFilter}, basierend auf den Abgaben der Autorin zu Übung vier, filtert mit Eingabe einer Liste GameSales und des RegionTypes die Datenpunkte nach den je gewünschten Regionsauswahlen für die Achsen. 
Sie wird zur Interaktion innerhalb des Plots, zur freien Auswahl der Achsen und korrekten Anzeige der entsprechenden Datenpunkte benötigt.
Die Anwendung wird in \textit{MainScatterParallel} vorgenommen und beschrieben.

In \textit{point} wird die Positionierung der Punkte sowie deren Beschriftung für die Weiterverwendung in \textit{Scatterplot} festgelegt.
Der \textit{scatterplot} ist für das Zeichnen des Plots zuständig. 
Mit der lokalen Funktion \textit{pointsXY} wird die Variabilität der Achsen und damit X- und Y-Werte der Punkte ermöglicht.
%Dies ist einzig in der Abgabe der Autorin zu Übung vier umgesetzt, sodass diese als Grundlage für die lokale Funktion dient. 
%In weiteren lokalen Funktionen wird die Abbildung bzw. Umrechnung der Daten auf SVG berechnet.
Im SVG des \textit{scatterplot} wird die Positionierung der X- und Y-Achse, ihrer Beschriftungen und der Punkte festglegt. 
Letzteres wird durch die lokalen Funktionen \textit{xScaleLocal}, \textit{yScaleLocal} und \textit{pointsXY} sowie der globalen Funktion \textit{point} erreicht.
Der \textit{scatterplot} basiert auf den Abgaben der Autorin zu Übung 1.4 sowie Übung vier für die lokale Funktion \textit{pointsXY}. 

Global wird wiederum das \textit{cssPoint} definiert.
Zuletzt werden wie zuvor allgemeine, konstant für den Plot geltende Einstellungen, basierend auf Übung eins bis vier, global definiert. 

Die Implementierung des Scatterplots verlief in weiten Teilen problemlos, wenn auch langsam durch ständiges Überprüfen der richtigen Datentypen.
Die Umsetzung des Mappings und Filterns nach Null-Werten stellte wie beim \textit{ParallelPlot} ein Problem dar, wobei die Lösung vereinfachend größtenteils von dort übernommen werden konnte. 
Weiterhin musste beachtet werden, dass die Achsen des Plots frei wählbar, also variabel definiert sind, anders als in den meisten Übungsserien.
Mit entsprechender Zeit- und Denkinvestition war auch dies überwindbar.

\subsection{MainScatterParallel.elm}
Auch das Modul \textit{MainScatterParallel} definiert zunächst die Importe. 
Aufgrund der besseren Erkenntlichkeit aus welchen Modulen welche Funktionen stammen, werden keine Funktionen dezidiert importiert, sondern nur das allgemeine Modul. 

Es wird die gesamte Elm-Architektur verwendet, da das Modul den \textit{Scatterplot} und die \textit{ParallelPlot} ausführen und alle drei Visualisierungen verbinden soll.
Hierzu wird das Programm zunächst wieder in der \textit{main}-Funktion definiert, sowie die \textit{subcritpions-} und die \textit{init}-Funktion implementiert.

Danach folgt die Definition des \textit{type Model} und \textit{type Msg}. 
%Diese könnten zwar auch in Data definiert werden, verweilen aber zugunsten der Übersichtlichkeit und schnelleren Verständnisses des Codes in diesem Modul.
In \textit{type Model} werden die drei Varianten Error, Loading und Success beschrieben, die in den entsprechenden Fällen angezeigt werden. 
Die Variante Success ist gleichzeitig ein Record aus den dort beschriebenen Feldern. 
Wichtig zu beachten ist, dass die initialen Werte erst in der \textit{update}-Funktion hier hinein geschrieben werden.
Im Feld data können die geladenen und decodierten Daten als Liste vom Typ GameSales gespeichert werden. 
Zur Ermöglichung der Interaktionen gibt es weiterhin Felder für das Genre, die Achsen eins bis fünf, die Namen der Achsen eins bis fünf, die X- und Y-Achse und den gewünschten Plot. 
Für die Achsen des Scatterplots und Parallelen Koordinaten Plots sowie den Plot an sich sind die entsprechenden CustomTypes definiert.
Der \textit{type Msg} definiert die Varianten für die Nutzung in \textit{update} zur Überschreibung des Models. 
Es gibt eine Variante zum erfolgreichen Laden der Daten und damit der Initialisierung des Plots sowie für jeden möglichen Wechsel von Achsen in den Plots sowie den Wechsel des Plots an sich.

In der \textit{update}-Funktion werden zuerst die initial anzuzeigenden Attributwerte mit Rücksicht auf die zuvor festgelegten Datentypen definiert und nach Anwendung der Dekodierung in Success gespeichert. 
Weiterhin werden für alle Interaktionen mit dem Model die Varianten von \textit{Msg} eingefügt und beschrieben, wie bei einer Änderung dieser Variante durch Klick auf deinen Button das Model überschrieben wird.

Nachfolgend werden alle benötigten Button-Funktionen zur Überschreibung des Models bei Klick auf eine Variante implementiert.
Wichtig ist die Nutzung der in \textit{Data} definierten Umwandlungsfunktionen für alle außer den \textit{buttonGenreType} von Strings, welche die Buttons benötigen, zu den entsprechend für das Model definierten Datentypen. 
Die Änderungen können dann in der Variante des \textit{Msg}-Typen gespeichert und durch die \textit{update}-Funktion das Model überschrieben werden.
%Bild Button ParallelPlot
Die Buttons basieren auf den Abgaben der Autorin aus Übung vier.

Die Funktion \textit{filterGenre} beruht auf den Abgaben der Autorin zu Übung vier und wird folgend im \textit{view} angewandt. 
Dem Anwender wird dadurch eine Filterung nach dem Attribut Genre ermöglicht.
%Im Prinzip sorgt diese Funktion, wenn später im \textit{view} angewendet, dafür, dass ausgewähltes Genre und angezeigtes Genre übereinstimmen, also Gleichheit besteht.

In der \textit{view}-Funktion werden die verschiedenen Fälle des Models zur Ansicht behandelt, wobei hier nur auf den Erfolgsfall näher eingegangen wird. 
Zunächst wird eine \textit{let-in}-Konstruktion auf erster. äußerer Ebene erstellt. 
In lokalen Funktionen wird \textit{gameSalesData} sowie die Anzahl der Spiele insgesamt beschrieben.
Zusätzlich wird hier der globale \textit{filterGenre} zur Filterung der Daten nach gewünschtem Genre angewandt.
Auch die Länge dieser Liste wird berechnet. 
Damit nun zwischen den Plots gewechselt werden kann, werden die verschiedenen Varianten bzw. Fälle des Datentyps \textit{PlotType} wie im Record der Variante \textit{Success} des Models aufgeführt im \textit{in}-Teil angelegt. 
Je nachdem, welcher Plot mittels Button ausgewählt wird und wie das Model durch das \textit{update} überschrieben wurde, wird einer der drei Fälle angezeigt. 
Der Zustand des Models mit dem gefilterten Genre soll bestehen bleiben, sodass eine Interaktion mit dem Genrebutton in einer Visualisierung auch in der anderen angewandt und übernommen wird.
Dazu dient zum einen die zuvor beschrieben Variantendefinition im \textit{in}-Teil der äußeren \textit{let-in}-Ebenenkonstruktion sowie jeweils eine weitere \textit{let-in}-Konstruktion in den Falldefinitionen.
Diese befinden sind auf zweiter, innerer Ebene, sodass die in der äußeren Konstruktion berechneten Zustände, sprich Genreauswahlen, auch hier gelten und bei einem Wechsel des Plots übernommen werden. 
In den lokalen Funktionen der inneren Konstruktion der Plots muss dazu die lokale \textit{gameSalesDataFiltered} der äußeren Ebene angewandt werden. 
Würde in jeder der inneren Ebenen für jeden Plot die Filter nach Genre neu definiert werden und lokal angewandt statt in der übergeordneten Ebene, würden die Einstellungen repsektive der Zustand des Models nicht übernommen werden.
%Bild Code

Im ersten Case wird der ParallelPlot behandelt. 
Lokal wird die \textit{assignmentAndReduce}-Funktion auf die gesamten Daten sowie auf die in der äußeren Ebene nach Genre gefilterten Daten angewandt.
In \textit{multiDimFunction} wird \textit{multiDimenData} letztlich durch Übergabe der entsprechenden bereinigten und gefilterten Daten genutzt. 
Im \textit{in}-Teil der inneren Ebene wird die Struktur und das Design der Seite festgelegt.
Für eine angenehmere und weniger ablenkende Ansicht werden die Farben den Grüntönen in den Plots angepasst. 
Zur Direktion des Auges über die Seite und verbesserten Erkennung von wichtigen Abschnitten wie der Auswahl der Plots, des Genres und der Achsen werden Ränder und verschiedene Grüntonabstufungen genutzt.
Neben Informationstexten und Überschriften werden hier auch die Buttons als Dropdown-Menü zum Wechsel des Plots, zur Änderung des Genrefilters sowie der Anpassung der Achsen eingefügt.
Eine transparente Anzeige der gewählten Einstellungen und verbleibenden Videospiele wird hinzugefügt.
Schließlich wird dem \textit{scatterplotParallel} das \textit{cssParallel} sowie zwei Konstanten für die Höhe und die Aspect Ratio übergeben bevor die zuvor lokal berechnete \textit{multiDimFunction} folgt. 
Zuletzt wird noch ein Text mit Link direkt zurück zur TreeHierarchy in einem neuen Tab angeboten.

Im zweiten Case wird der Scatterplot behandelt. 
Das Vorgehen mit den lokalen Funktionen zur Anwendung der Filter und des Mappings ist ähnlich wie zuvor.
Durch die lokalen Funktionen \textit{valuesX} und \textit{valuesY} werden dem \textit{regionFilter} die nach Genre gefilterten Daten als Liste von GameSales sowie die RegionTypes der X und Y-Achse übergeben.
Somit wird der Filter nach Regionen angewandt und die Interaktion mit freier Auswahl der Achsen kann funktionieren.
Der \textit{in}-Teil der Plotvariante Scatterplot ähnelt dem des ParallelPlots stark und unterscheidet sich nur in den Informationstexten sowie der Auswahl der Achsen.
Zuletzt wird dem \textit{scatterplot} der \textit{cssPoint}, die zuvor berechneten \textit{gameSalesDataCleared} sowie die \textit{valuesX} und \textit{valuesY} übergeben.
Zur Beschriftung der Achsen werden diese mittels der Umrechnung von RegionType zu String als String hinzugefügt.

Da die TreeHierarchy in \textit{TreeHierarchy.elm} implementiert ist, muss für diese Plotvariante nur das Html inklusive Link zum eigentlichen Plot des Baumdiagramms im selben Design wie zuvor definiert werden.
%Da für das Baumdiagramm der Genrefilter irrelevant ist, muss er weder hier noch in TreeHierarchy angewandt werden. 

Durch die zuvor in getrennten Anwendungen implementierten Visualisierungen war ein grundlegendes Zusammenfügen dieser durch stückweises Kopieren unkompliziert umsetzbar.
Besonders die Erstellung einer einheitlichen \textit{update}-Funktion war jedoch zeitlich durch die Menge an Interaktionsmöglichkeiten aufwändig.
Weiterhin schwieriger war die Implementierung der des Dropdown-Menüs für den ParallelPlot, da die Buttons zunächst je einzeln vorlagen. 
Durch Nachdenken und Ausprobieren vor allem mit der Funktion \textit{multiDimenData} und verschiedenen Konvertierungsfunktionen konnten sie doch wie gewünscht realisiert werden.
Zur Überprüfung immer wieder zeitaufwendig das \textit{update} angepasst werden.
Zunächst wurden beide Plots als untereinander auf einer Seite gezeichnet implementiert.
Entsprechend war ohne Plot-Selektor nur eine \textit{let-in}-Kontruktion nötig und die Filterung wurde für beide übernommen, was nicht dem Gewünschten entsprach.
Nach schrittweisem Hinzufügen der Auswahl des Plots und allen zugehörigen Codeteilen, wurde durch Ausprobieren entdeckt, dass eine Übernahme des Modelzustands durch die Schachtelung der \textit{let-in}-Konstruktionen im \textit{view} und der Anwendung des Variantenwechsels für den Plot mittels \textit{PlotType} möglich war. 
Hier war es im Verlauf kompliziert, den Überblick über die Schachtelungen sowie außen und innen definierten lokalen Funktionen zu behalten.

%Beschreiben Sie die Implementierung ihrer Visualisierungsanwendung in Elm. Stellen die Gliederung ihres Quellcodes vor. Haben Sie verschiedene Elm-Module erstellt. Was war aufwändig umzusetzen, was ließ sich mit dem vorhanden Code aus den Übungen relativ einfach umsetzen? 
%Wie sieht die Elm-Datenstruktur für das Model aus, in dem die verschiedenen Zustände der Interaktion gespeichert werden können.
\section{Anwendungsfälle}
Im folgenden Kapitel wird ein möglicher Anwendungsfall der implementierten Visualisierungen dargestellt.
Dabei werden diese aus der Sicht eines mittleren Managers des Publishers \textit{505 Games} betrachtet, der sie und die Erkenntnisse aus ihnen dem oberen Management des Publishers sowie Stakeholdern des Unternehmens präsentiert. 
Daraus sollen Indikatoren für potentielle Investitionen in neue Videospiele bzw. Fortsetzungen vorhandener Titel eines Genres gefunden werden. 
Kann ein solcher Indikator für Potenzial gefunden werden, so kann entschieden werden, ob weitere detailliertere Marktstudien in Auftrag gegeben werden sollen. 
Zudem können Vorentscheidungen zur Aufstellung eines neuen Videospiels im Markt getroffen werden, wobei zwischen einer breiten, globalen Aufstellung oder der Konzentration auf bestimmte Märkte resp. Regionen entschieden werden kann.

%Präsentieren sie für jede der drei Visualisierungen einen sinnvollen Anwendungsfall in dem ein bestimmter Fakt, ein Muster oder die Abwesenheit eines Musters visuell festgestellt wird. Begründen sie warum dieser Anwendungsfall wichtig für die Zielgruppe der Anwenderinnen ist. Diskutieren sie weiterhin, ob die oben beschriebene Information auch mit anderen Visualisierungstechniken hätte gefunden werden können. Falls dies möglich wäre, vergleichen sie die den Aufwand und die Schwierigkeiten ihres Ansatzes und der Alternativen. 
\subsection{Anwendung Visualisierung Eins}
Zuerst lädt das explizite Baumdiagramm. 
%Dank Überschriften und Unterüberschriften ist eine inhaltliche Orientierung auf der Website gegeben.
%Der Informationstext hilft beim Verständnis des Diagramms.
Mittels farblicher Unterschiede der einzelnen Websitebereiche sowie der Abgrenzung der Container durch Umrandungen wird das Auge der Betrachter schnell und unbemerkt auf die wichtigen Steuerungselemente sowie die eigentliche Visualisierung gelenkt.
Zu diesem Zweck wird mit Ausnahme des Navigationscontainers mit \textit{Just Noticable Differences} gearbeitet.
Sollte eine Person mit Farbsehschwäche unter den Präsentationsteilnehmern sitzen, so ist dies durch Abstufungen der Grüntöne auch im Kontrast auf der Website und in den Visualisierungen kein Problem. 
Falsche Farberkennungen spielen für das Verständnis keine Rolle.

Im ersten Ausschnitt des Baumdiagramms kann der Präsentator den Teilnehmern einen Überblick über die Publisher gewähren sowie die eigene Position im Baum aufzeigen.
Im zweiten, detaillierteren Ausschnitt wird den Managern und Stakeholdern dargestellt, welche Genres sie mit \textit{505 Games} bedienen sowie die darunter kategorisierten Spiele.
Insgesamt bieten sie fünfzehn Videospiele in acht Genres an, wobei \textit{Adventure}, \textit{Racing} und \textit{Shooter} die meisten Spiele haben.
Dank der Einblendungen der Titel und Genres beim Hovern über die sich grün färbenden Knoten in beiden Diagrammen sowie das Scrollen im zweiten Diagramm sind konkrete Ein- und Überblicke effektiv möglich. 
Es wird entschieden, die Genres \textit{Action}, \textit{Platform} und \textit{Sports} aufgrund der bisher eher geringen Ausprägung im eigenen Unternehmen näher zu betrachten. 

Aus dem Baumdiagramm ergibt sich durch die hohe Anzahl an anbietenden Publishern vor allem eine Konkurrenz im Genre \textit{Action}. 
Durch die Struktur und Positionierung der Knoten ist schnell erkennbar, dass hier vor allem \textit{Activision}, \textit{Capcom}, \textit{Ubisoft} und \textit{Warner Bros.} durch ein hohes Spieleangebot die stärkste Konkurrenz zu sein scheinen. 
In diesem Genre kann Potenzial durch viele Käufer liegen, da viele Publisher hier Spiele anbieten. 
Gleichzeitig gibt es viel Konkurrenz, wodurch eine Sicht auf mögliche Korrelationen in den Regionen der Welt und die entsprechend beste Positionierung im Markt wichtig ist.
\textit{Sports} und vor allem \textit{Platform} haben weniger Konkurrenz, wobei gerade ersteres durch Publisher wie \textit{EA Sports} und \textit{2K Sports} dominiert wird.

Aufgrund dieser ersten Übersicht entscheidet der Präsentator, dass nachfolgend der Fokus auf die Gernes \textit{Action} und \textit{Sports} gelegt wird.
Mit Klick auf den Link in der Navigation gelangt der Betrachter zu den detaillierteren Visualisierungen.

Die Informationen hätten auch durch ein implizites Baumdiagramm ermittelt werden können. Aufwand und Schwierigkeit der Implementierung wären im Vergleich leicht erhöht, das Verständnis bei den potenziell ungeschulten Betrachtern, vor allem Stakeholdern, gerade in kurzen Momenten einer Präsentation aber leicht gemindert. 
Baumdarstellungen anderer Anordnung, z.B. radialer, wären auch möglich, würden gut verstanden werden, zögen aber deutlich erhöhten Aufwand und Schwierigkeiten der Implementierung mit sich. 
%Somit sind sie für eine Voranalyse der Verkaufszahlen im Videospielmarkt wie hier angenommen, zu aufwändig.

\subsection{Anwendung Visualisierung Zwei}
In dieser Visualisierung eines Parallelen Koordinaten Plots, kann der mittlere Manager in seiner Präsentation den oberen Managern und Stakeholder einen detaillierteren Blick auf die Verkaufszahlen der Videospiele eines Genres in den verschiedenen Regionen der Welt bieten.
Der Blick der Betrachter wird wie zuvor beschrieben gelenkt.
Zusätzlich finden sich wieder Erklärungen zu Einheiten und Darstellungsweisen auf der Webseite.

Der Präsentator wählt nun das zuvor festgelegte Genre \textit{Action} aus und kann den Parallelen Koordinaten Plot mit den voreingestellten Achsenbelegungen und den gefilterten Videospielen als Datenpunkte betrachten. 
Schnell fällt der beinahezu allen Spielen sichtbare Abstieg der Verkaufszahlen in Japan auf. 
Um folgend Zusammenhänge zwischen den anderen Regionen besser erkennen zu können, wird die Achsenbelegung wie im folgenden Bild gezeigt ausgewählt.
%Bild
Durch die Röntgenstrahlen nachempfunde Darstellungsweise lässt sich nun ein Muster erkennen. 
Zum einen befinden sich der Großteil der Videospiele im unteren drittel der Verkaufszahlen in allen Regionen, was durch die Überlappungen der Datenpunkte und damit kräftigeren Grüntöne besser erkennbar ist.
Einzig \textit{Assassin's Creed: Unity} und vor allem \textit{Grand Theft Auto V} von \textit{Rockstar Games} stechen heraus. 
Wenn sich Spiele gut in Nordamerika verkauften, so verkauften sie sich in Europa in ähnlicher Weise, genauso wie im Rest der Welt und global.
Dies wird auch durch unterschiedliche Achsenbeledungen bestätigt. Eine Korrelation über alle Regionen hinweg scheint gegeben. 
Einzig Japan bildet aus unbekannten Gründen eine Ausnahme, da die meisten Spiele hier wie anfangs erwähnt nicht so gut verkauft werden, wenn nicht sogar gar nicht. 
Gleichzeitig schneiden einige wenige Spiele, die in den anderen Regionen durchschnittlich verkauft werden, in Japan vergleichsweise gut ab. 

Durch die Korrelation scheint eine Konzentration in diesem Genre auf eine bestimmte Region wenig Sinn zu ergeben, sodass eine breite Aufstellung und Nutzung der Beeinflussung nützlich ist.
Gleichzeitig deutet die ansatzweise Bildung eines Clusters im unteren Drittel der Verkaufszahlen in alles Regionen auf starke Konkurrenz im Genre hin. 
Diese wird verstärkt durch die identifizierten Ausreißer, welche wie alle anderen Titel durch Hovern über den Datenpunkt gut ablesbar sind wie obiger Abbildung zu sehen.
Ein neuer Titel sollte also sehr überzeugend sein oder eventuell doch in einem anderen Genre platziert werden. 
Sollte nach einer weiteren Bestätigung der Erkenntnisse im Scatterplot die Entscheidung zu weiteren Marktstudien in dem Genre getroffen werden, müsste anhand weiterer Einflussparameter unter anderem der Grund für die hohen Verkaufszahlen geprüft, sowie die Sättigung des globalen Marktes durch diesen Titel genauer analysiert werden.

Im Gegensatz zu \textit{Action} gibt es im Genre \textit{Sports} keine Ausreißer über alle Regionen hinweg und die Verkaufszahlen verteilen sich stärker über die Skala. 
Japan mit fast nur Verkaufszahlen von null bleibt bestehen. Eine Begründung kann die Visualisierung nicht liefern.
%Bild
Deutlich erkennbar ist jedoch, dass einige Spiele in Nordamerika besser verkauft werden als in Europa, im Rest der Welt und global aber wieder ähnlich, wenn auch etwas schlechter, als in Nordamerika. 
Gleichzeitig verkaufen sich Spiele, die in Europa gut verkauft werden, in Nordamerika etwas schlechter, im Rest der Welt und global aber wiederum besser. 
Daraus ergibt sich eine erste Tendenz, dass eine Konzentration auf eine der Regionen \textit{North America} oder \textit{Europe} sinnvoll ist, um wiederum die positiven Beeinflussungen dieser jeweiligen Regionen auf die restlichen Regionen zu nutzen. 
Sie wird im Scatterplot überprüft.

Der Vergleich mehrerer Dimensionen resp. mehrerer Attribute eine Videospiels wäre auch durch bspw. Icontechniken möglich. 
Diese Methode wäre jedoch schwerer umzusetzen und aufwändiger zu analysieren. 
Der Grund liegt in der Darstellung aller Dimensionen eines Videospiels auf einem Icon und einer entsprechend erschwerter Erkenntnis der Zusammenhänge und Muster über die Spiele hinweg. 
Auch Außreißer sind mit dieser Technik nicht so gut erkennbar.

%Wenn bspw. eines seiner Spiele in allen oder einzelnen Regionen nicht gut verkauft werden konnte, ein Spiel eines Konkurrenten in diesen Regionen jedoch gut, dann deutet das auf eine starke Konkurrenz hin. 
%Sind alle oder die meisten Spiele der Konkurrenz in allen oder einzelnen Regionen besser verkauft worden, als das Videospiel von Publisher X, dann deutet dies auf einen gesättigten Markt des Genres mit starker Konkurrenz hin, in das sich möglicherweise weitere Investititonen nicht lohnen. 
\subsection{Anwendung Visualisierung Drei}
Um zur dritten Visualisierung, dem Scatterplot, zu gelangen, wählt der mittlere Manager in seiner Präsentation den Plot im Drop-Down-Menü aus.
Die zuvor getroffene Auswahl des Genres, zuletzt \textit{Sports} bleibt bestehen und er wählt in einem weiteren Drop-Down-Menü die Belegung der Achsen auf \textit{North America} und \textit{Europe}.

Im Scatterplot werden die zuvor erkannten Cluster in den Verkaufszahlen im Vergleich von Nordamerika zu Europa deutlicher. 
%Bild Sports
Während bei einigen Spielen zwischen beiden Regionen ein deutlich positiv korreliertes Cluster sichtbar ist, scheinen die anderen Spiele ein Cluster ohne Korrelation zwischen den Regionen zu bilden.
Dies bestätigt und verdeutlicht die Erkenntnisse aus der zweiten Visualisierung, dass sehr gute Verkaufszahlen von einigen Sportvideospielen in Nordamerika keinen Einfluss auf die Verkäufe in Europa haben. 
%Dort verkaufen sie sich vergleichsweise schlecht. Jedoch ist durch die Ansammlung der Datenpunkte in einer waagerechte, bzw. in umgekehrter Anordnung der Achsenbelegung senkrechten, Linie auch keine negative Korrelation ersichtlich.
Die Titel im Cluster der positiv korrelierten Spiele scheinen sich jedoch gegenseitig positiv zu beeinflussen. 
%Es kann durch etwas geringeren Verkaufszahlen in Nordamerika vermutet werden, dass die europäischen Verkaufszahlen die nordamerikanischen positiv beeinflussen. 
Verweist der Präsentator mittels Hovern über die Datenpunkte auf die Titel der Videospiele, wird deutlich, dass diese Cluster mutmaßlich durch die unterschiedliche Popularität der Sportarten in beiden Regionen entsteht. 
So ist Football in Europa nicht so bekannt oder beliebt, Fußball aber in Nordamerika schon, wenn auch leicht weniger als in Europa.

Zuletzt interessieren sich die Teilnehmer der Präsentation für die Überprüfung der Erkenntnisse zum Genre \textit{Action}, sodass die Filterung entsprechend angepasst wird. 
In allen möglichen Kombinationen der Achsenbelegung, abgesehen von Japan, bestätigt sich deutlich erkennbar die leicht bis vollständig positive Korrelation zwischen den jeweiligen Verkaufszahlen der Regionen.
Im Scatterplot sind die einzelnen Datenpunkte etwas besser erkenn- und zur näheren Betrachtung auswählbar. 
Auch der zuvor erkannte Ausreißer sticht deutlich heraus. 
Zusätzlich bestätigt sich die nicht vorhandene Korrelation zwischen den jeweiligen Regionen und Japan.
Sollte während der Präsentation der Wunsch nach einer gleichzeitigen Betrachtung des Baumdiagramms zur Übersicht der Publisher aufkommen, so kann mittels des Links am jeweiligen Seitenende in einem neuen Fenster dieser geöffnet werden und per Multitasking betrachtet werden.

Eine sinnvolle Alternative zu Scatterplots ist nicht gegeben, da sie die beste und am leichtesten zu verstehende Darstellung von Beziehungen zweier Attribute eines Datenpunktes ist. 
%Aufgrund der Zweidimensionalität der Verteilungen würden auch Q- und QQ-Plots mit nur einer visualisierten Dimension gegen f-Werte bzw. gegeneinander zum Erkennen von Verschiebungen nicht die gewünschten Ergebnisse liefern.

Durch die Erkenntnisse aller Visualisierungen können sich die oberen Manager und wichtigsten Stakeholder von \textit{505 Games} nun dazu entscheiden, sich bei weiteren, (kosten-) intensiven Marktstudien mit detaillierteren Visualisierungen und weiteren Attributen auf das Genre \textit{Sports} zu konzentrieren. 
Dort sind sie selbst noch nicht stark präsent, es gibt weniger und in der Masse weniger starke Konkurrenz als bei \textit{Action} sowie keine extremen Ausreißer. 
Weiterhin müssen nicht die Abhängigkeiten zwischen allen Regionen betrachtet werden, sodass eine Fokussierung und Abschöpfung eines Marktes Sinn ergibt. 
Die Fokussierung auf den europäischen oder nordamerikanischen Markt ist möglich und für ein kleineren Publisher wie \textit{505 Games} sinnvoller. 
Durch die Ausrichtung auf den europäischen Markt kann \textit{505 Games} jedoch die positive Korrelation mit dem nordamerikanischen Markt für seine Verkaufszahlen nutzen und damit auch ihn bedienen.
Untersucht werden sollten künftig auch die Einflüsse für die vergleichsweise guten Verkaufszahlen für das eigene Spiel \textit{Rocket League}, um über eine mögliche Fortsetzung zu entscheiden.
 
\section{Verwandte Arbeiten}
In diesem Kapitel wird eine knappe Literatursuche nach ähnlichen Anwendungen zu Videospielverkäufen oder Verkaufsdaten im Allgemeinen im Bereich der Informationsvisualisierung und Visual Analytics durchgeführt. 

Der erste zu diskutierende Artikel ist \textit{VizInteract: Rapid Data Exploration Through Multi-touch Interaction with Multi-dimensional Visualizations} von Chakraborty und Stuerzlinger.\cite{Chakraborty.2021}
In diesem Artikel wird \textit{VizInterct} vorgestellt und getestet. 
Es dient der Multi-Touch Interaktion, um multidimensionale Datenvisualisierung schneller und einfacher konstruieren und mit ihr interagieren zu können.
Es sind verschiedene Datenvisualisierungen implementiert, die mittels Verschieben und Übereinanderschieben zu neuen Plots resultieren können. 
So können laut der Autoren z.B. zwei orthogonale Histogramme durch Übereinanderschieben einen Scatterplot kreieren. 
Forschungsziel des konkreten Artikels die Beobachtung des Nutzerverhaltens der Anwendung einfacher touchbasierter Interaktionen.

Als Anwendungsfall wird für die Tester unter anderem ein Unterdatensatz der \textit{Video Game Sames 2019} bereitgestellt, mit dem sie Visualisierungen mittels des Tools erstellen und damit vorgefertigte Analysefragen beantworten sollen.
Der Datensatz ähnelt dem hier verwendeten, gleicht sich jedoch nicht. 
In \textit{VizInteract} sind Histogramme, Scatterplots, Parallele Koordinaten Plots, Scatterplotmatrizen und Sterndiagramme möglich. 
Somit stimmen auch zwei der im Projekt verwendeten Visualisierungstechniken mit denen im Artikel überein. 

Gemeinsamkeiten zwischen der Anwendung und Implementierung im vorliegenden Projekt und dem Artikel liegen in der Möglichkeit, Notwendigkeit und Umsetzung von Interaktionen für sinnvolle Visualisierungen, z.B. mittels Filtern.
Weiterhin werden in beiden Fällen Scatterplots und Parallele Koordinaten eingesetzt und in den den Testern gestellten Aufgaben ähnlich angewandt wie im Anwendungsfall des Projektes. 
Durch den Umfang von \textit{VizInteract} und die Tests im Artikel zeigt sich wiederum die Wichtigkeit, die die Autoren den Visual Analytics beimessen.
Klar ersichtlich ist die Gemeinsamkeit der Nutzung der Videospielverkäufe mit einem sehr ähnlich aufgebauten Datensatz sowie Visualisierungen.
Unterschiede ergeben sich aus den weiteren Visualisierungsmöglichkeiten des Tools, der Bedienbarkeit mit Touch-Oberflächen sowie der fortgeschrittenen Interaktion mittels Komposition der Techniken. 
Im Artikel wird das Tool anders als hier vorliegend mittels eines Anwendungsfalles auf seine Hauptmerkmale geprüft und seine Handhabung fokussiert.

Als zweites sei \textit{Intelligent Visual Analytics Queries} von Hao et al. genannt.\cite{Hao.2007}
In diesem Artikel möchten die Autoren Analysten mittels ihrer \textit{Intelligent Visual Analytics Query} unterstützen.
Der Fokus liegt auf großen multidimensionalen Datensätzen, in die durch das Tool Einsicht in komplexe Muster, Phänomene und Ausreißer erlangt werden soll.
Die angestrebte Anwendung beschreibt einen Analysten, der in einer noch unübersichtlichen Visualisierung der Daten einen Interessensbereich sowie die dazugehörigen Attribute auswählt.
Anschließend sollen durch automatische analystische und visuelle Analysemethoden Charakterisiken und Beziehungen zu anderen Attributen und Datenpunkten identifiziert werden.
Der im Artikel genutzte Anwendungsfall bezieht sich auf Verkaufszahlen von Produkten und Kundenkaufverhalten, da Verkaufsanalysten Produktverkäufe und Promotionen laut der Autoren korrelieren wollen. 

Das vorgestellte Tool nutzt eine tabellenartig angelegte visuelle Karte, bei der in den Zeilen die Gruppen an Dimensionen und in den Spalten die Datenintervalle gespeichert werden.
Die Farbe kennzeichnet den Attributwert für den Datenpunkt. So können Korrelationen und Ähnlichkeitern mehrerer Attribute erkannt werden.
Weiterhin wird wie in der hier vorliegenden Arbeit der Parallele Koordinaten Plot visualisiert. 
Es können interessante Untergruppen von Daten ausgewählt und die paarweise Korrelation der gewählten Attribute berechnet werden. 
Die Achsen werden schließlich so angeordnet, dass hochkorrelierte Attribute nah beieinander sind. 
Auch Scatterplots werden genutzt, wobei diese hier so angeordnet werden, dass weiterhin multidimensionale statt zweidimensionale Daten abgebildet werden können.

Gemeinsamkeiten bestehen in der Auswahl der Visualisierungstechniken und der Anwendung dieser zum Erkennen von Mustern und Korrelationen in großen multidimensionalen Datensätzen zu Verkaufszahlen.
Zusätzlich wird die Nützlichkeit dieser Techniken und der Visual Analytics an sich für die Auswertung und das Verständnis von Verkaufsdaten ähnlich groß eingeschätzt.
Auch die Nutzung und Bewertung der Existenz von Interaktionsmöglichkeiten in den Visualisierungen durch Filter deckt sich mit denen der Autorin der vorliegenden Arbeit.
Unterschiede befassen sich mit dem Einsatz von Scatterplots für mehr als zwei Dimensionen.
Trotz der Ähnlichkeiten der Parallen Koordinaten Plots ist die Anordnung der Achsen hier verändert und nimmt dem Nutzer Arbeit ab. 
So muss dieser nicht selbst die Achsen variabel aussuchen, um Muster und Korrelationen besser zu erkennen. 
Gleichzeitig fehlt ihm diese Flexibilität bei Hao et al.
Wie schon im vorherigen Artikel wird weniger die Beantwortung einer Frage aus dem Anwendungsfall heraus behandelt, sondern das Tool allgemein vorgestellt.

%Führen sie eine kurze Literatursuche in der wissenschaftlichen Literatur zu Informationsvisualisierung und Visual Analytics nach ähnlichen Anwendungen durch. 
%Diskutieren sie mindestens zwei Artikel. Stellen sie Gemeinsamkeiten und Unterschiede dar.

\section{Zusammenfassung und Ausblick}
In dieser Arbeit wurde mittels drei Visualisierungstechniken eine erste Marktanalyse der Verkaufszahlen des Videospielemarktes für die Konsole \textit{XBoxOne} durchgeführt. 
Für die Zielgruppe des mittleren und oberen Managements als auch eingeschränkt der Stakeholder der Videospielverlage, hier Publisher, wurden die vorliegenden Daten visuell aufbereitet. 
%Zum einen wurde ein Überblick über den Markt, das eigene Verlagshaus und die Konkurrenz ermöglicht, der in Präsentationen und zur wiederkehrenden Ansicht und Eigenanalyse der Zielgruppen genutzt werden kann.
%Zum anderen wurde mittels der Visualisierungen und sich zeigenden Muster und Korrelationen in den Verkaufszahlen zwischen den Regionen pro Genre eine informiertere Entscheidungunterstützung über potenzielle künftige Investitionen in neue Titel oder Fortsetzungen in den entsprechenden Genres bzw. tiefere und teurere Marktstudien hierzu ermöglicht.
%Die Visualisierungstechniken wurden vom Groben zum Detaillierten angeordnet. 
Durch die Implementierung eines expliziten Baumdiagramms konnte die gewünschte hierarchische Übersicht über den Markt sowie die Zusammenhänge zwischen Publisher, Genre und Titel umgesetzt werden.
Gleichzeitig konnte durch diese Technik die Konkurrenz betrachtet und abgeschätzt werden, welche Genres durch die quantitative Anzahl der Spiele und deren Publisher Potenzial bieten und im Folgenden näher betrachtet werden sollten.
Mittels eines Parallelen Koordinaten Plots zur Darstellung mehrdimensionaler Daten konnten auch durch den Einsatz einer Art Röntgenstrahlentechnik Muster in den Verkaufszahlen von Videospielen eines Genres über mehrere Dimensionen, also Regionen, hinweg erkannt werden.
Zuletzt konnten in einem klassischen Scatterplot die Erkenntnisse aus der vorherigen Visualisierung überprüft werden. 
Korrelationen zwischen zwei gewählten Regionen sowie Ausreißer und die Bildung von Clustern konnten durch diese Visualisierung detaillierter dargestellt und analysiert werden.
Für ein verbessertes, komfortables und schnelles Anwendungserlebnis sowie Effektivität und Effizienz wurde die Auswahl der Genre mittels Drop-Down Menü interaktiv gestaltet sowie beim komfortablen Umschalten der Visualisierungen ebenso mittels Drop-Down Menü erhalten.
Aus demselben Grund wurde auch ein Beibehalten der flexiblen und individuellen Achsenauswahl für den Parallelen Koordinaten Plot sowie den Scatterplot integriert.
%Zur nötigen Erkennung von Details zu Publishern, Titeln und konkreten Verkaufszahlen wurden interaktive Anzeigen dieser Informationen beim Hovern über die Datenpunkte in allen Visualisierungen ermöglicht.

Der Mehrwert der Visualisierungen liegt für die Zielgruppe vor allem im schnellen Verständnis und Überblick über den aktuellen Videospielmarkt mit Fokus auf Verkaufszahlen für die \textit{XBoxOne}.
Zusammenhänge und Muster zwischen den Regionen in den Genres können ohne Vorwissen im Bereich der Visual Analytics abgelesen, analysiert und präsentiert werden.
So bieten die Visualisierungen erste Erkenntnisse für die Zielgruppe bezüglich neuer Möglichkeiten und Chancen im Markt sowie Ansatzpunkte für grobe Strategieentscheidungen. 
Sie dienen als Entscheidungsunterstützung für die Beauftragung detaillierterer, fokussierter und teurerer Marktstudien für neue Investitionen.

Sinnvolle Erweiterungen der Daten wären der Einbezug von (Nutzer-)Kritiken und deren mögliche Auswirkungen und Korrelationen auf die Verkaufszahlen sowie zwischen den Regionen. 
Auch die Analyse bspw. auf Korrelationen nicht öffentlich zugänglicher Daten zu Entwicklungskosten und Umsätzen pro Spiel bieten Potenzial.
Dies ist jedoch Teil einer angesprochenen weiterführenden, detaillierten Marktstudie.
Interessant wäre eine weitere Unterteilung der Regionen, vor allem der unter \textit{Rest of World} zusammengefassten Regionen wie Afrika, Asien und Südamerika.
Bezüglich der Visualisierungstechniken kann eine Interaktion mit Klick auf einen Genreknoten im Baumdiagramm und die Anzeige des dazu passenden Parallelen Koordinaten- und Scatterplots Sinn ergeben.
Deutlich mehr Potenzial bieten jedoch Ansätze aus den \textit{Verwandten Arbeiten}. 
So ist ein automatisches Anordnen der Achsen je nach Korrelationsstärke im Parallelen Koordinaten Plots wie bei Hao et al. interessant.\cite{Hao.2007}
Zur Verbesserung der Interaktionen gerade auf der heutzutage vielfältigen Auswahl von Endgeräten ergibt eine Implementierung einer Komposition von Visualisierungen durch Zusammenschieben wie bei Chakraborty und Stuerzlinger Sinn.\cite{Chakraborty.2021}

Insgesamt konnten die durch die mögliche Anwendung entstandenen Anforderungen an die implementierten Visualisierungen gut umgesetzt werden.

\section*{Anhang: Git-Historie}

\printbibliography

\end{document}

