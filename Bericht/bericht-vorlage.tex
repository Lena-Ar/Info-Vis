\documentclass[usegeometry=true]{scrartcl}
\usepackage[ngerman]{babel}
\usepackage[T1]{fontenc}
\usepackage{lmodern}
\usepackage[utf8]{inputenc}
\usepackage{hyperref}
\usepackage{amssymb}
% Dimensionen bitte nicht ändern. 
\usepackage[left=2cm, right=2cm, top=2cm, bottom=2cm, bindingoffset=1cm, includeheadfoot]{geometry}
%Zeilenabstand bitte nicht ändern
\usepackage[onehalfspacing]{setspace}

\usepackage[backend=biber,style=numeric,]{biblatex}\addbibresource{literatur.bib}

\begin{document}
% ----------------------------------------------------------------------------
\subject{Projektbericht zum Modul Information Retrieval und Visualisierung Sommersemester 2022}
\title{Marktanalyse des Videospielemarktes}
\subtitle{Analyse und Visualsierung der Verkaufszahlen der Videospiele auf der Platform XBoxOne}
\author{Lena Arloth}% obligatorisch
%\date{10.9.2022}
\maketitle% verwendet die zuvor gemachte Angaben zur Gestaltung eines Titels
% ----------------------------------------------------------------------------
% Inhaltsverzeichnis:
%\tableofcontents
% ----------------------------------------------------------------------------
% Gliederung und Text:

\section{Einleitung}
\subsubsection {Motivation:} Der Spielemarkt ist groß, es steckt viel Geld dahinter und für die Entwicklung von neuen Spielen sind viele finanzielle Mittel nötig. Es braucht also eine gewisse Sicherheit, dass sich die Investitionen lohnen. Dazu braucht es eine Anaylse der verganenen Daten. XBox als Platform sehr beliebt, wenn auch hinter Playstation. Vergleichbarkeit der Spiele pro Platform, da zB XBox Spieler nicht mit Playstation-Spielern spielen können und Konsole nicht mit PC.
\subsubsection {Zielproblem:} Marktanalyse der Spieleindustrie der Platform XBox One aus Sicht der Publisher zur Ermittlung der Konkurrenz, der Erlangung von Kenntnissen über den Markt und über die eigenen Erfolge/den eigenen Standpunkt im Markt zur besseren Ausrichtung und Kontrolle des Unternehmens/Publishers anhand eines Marktüberblicks und genaueren Verkaufsfakten.
\subsubsection {Relevanz:} Jedes Unternehmen braucht eine Marktanalyse, so auch Spieleunternehmen/Publisher. Ein Publisher muss wissen, was er zukünftig mit guten Verkaufschancen entwickeln kann und in welchem Genre. Wo liegt der Schwerpunkt des Unternehmens.
\subsubsection {Fragestellungen:} 
\subsubsubsection {Überblick:} Welche Genres bedient ein Publisher aus seiner Sichtweise und wieviele Spiele bietet er in diesem Genre an? Wo liegt der Fokus/Schwerpunkt des Publishers? Wer ist die Konkurrenz in den verschiedenen Genres, der zB mehr Spiele in dem Genre anbietet?
\subsubsubsection {tiefere Marktanalyse:} Wie gut verkauften sich Spiele im Einzelnen eines Publishers eines Genres im Vergleich zu anderen Spielen des Genres? Wo gibt es Verbesserungspotenzial, wo gibt es Marktlücken, wo gibt es viel Konkurrenz? Gibt es Spiele in einem Genre eines Publishers, die sich nicht mehr lohnen?
\subsubsubsection {konkreter Blick auf jedes Spiel einzeln:} Wie verkauften sich Spiele des Publishers im einzelnen im direkten Vergleich auf den verschiedenen Märkten/Regionen der Welt?
\subsubsubsection {allgemein:} Welche Spiele waren gut, welche schlecht? Welche Spiele soll der Publisher anbieten in welchem Genre? Wo liegt Verbesserungspotenzial?

%Tipps zu Latex und Koma-Script für Hausarbeiten sind im \href{http://mirrors.ctan.org/info/latex-refsheet/LaTeX_RefSheet.pdf}{LaTeX Reference Sheet for a thesis with KOMA-Script} von Marion Lammarsch und Elke Schubert zusammengefasst. 
%Der Bericht fällt in die Kategorie von InfoVis-Paper, die Tamara Munzner Design Study nennt \cite{Munzner2008}: In der Einleitung sollen sie zuerst das Zielproblem beschrieben. Daraus sollen sie Fragestellungen motivieren, die mittels Techniken der Informationsvisualisierung beantwortet werden können. In dem Abschnitt direkt unter der Überschrift Einleitung sollen Sie nach einer kurzen Einleitung Fragestellungen und das Zielproblem motivieren und besschreiben. 

\subsection{Anwendungshintergrund}
Sinn der Marktanalyse in Unternehmen
Sinn der Vergleiche von Genres
Fakten zu Größe und Volumen des Spielemarkts ingesamt und in verschiedenen Regionen
Fakten XBoxOne und Platformen 
Fakten zu Anzahl Publisher und wieviele Spiele ein Publisher ungefähr bietet
Was die Anwendung/Visualisierung bringen soll

%Sie müssen genug Hintergrund bereitstellen, so dass die Lesenden sich ein Urteil bilden können, ob ihre Lösung funktioniert. Sie sollen die Lesenden jedoch nicht mit Anwendungsdetails so überschütten, dass der Fokus auf die Fragen zur Informationsvisualisierung untergehen. 

\subsection{Zielgruppen}
Publisher und dort oberes und mittleres Management/die Entscheider
Vorwissen: Kein detailliertes Vorwissen zu speziellen Visualisierungstechniken. Aber Vorwissen zu BWL-Analysen, da sie im Grunde Analysten sind. Kennen also aus ihrer täglichen Arbeit Diagramme, Scatterplots, Koordinatensysteme, Baumdiagramme, Zeitreihendiagramme und parallele Koordinaten. Eventuell auch Forces und Recursive Pattern. Kennen also nicht die ganz spezifischen Visualisierungstechniken/haben kein ganz spezifisches Vorwissen, sodass hier viel erklärt werden müsste, was nicht zielführend wäre. Das was sie kennen, vor allem Diagramme, Scatterplots, Bäume und Zeitreihendiagramme verstehen sie schnell, da sie es gut kennen und intuitiv damit arbeiten können. 
Zusätzlich müssen sie die Visualisierungen u.U. auch Business Partnern zeigen oder im eigenen Unternehmen dem weiteren Management. D.h. sie müssen immer wieder auch schnell auf den ersten Blick sehen können, was gezeigt ist und die Visualisierung sagen soll und es entsprechend schnell und intuitiv anderen erklären können. 
Informationsbedürnis: Harte Fakten. Was bietet der Publisher an? Denn großer Publisher = viele Entwickeler = viele Projekte/Spiele = eventuell braucht das Management erstmal eine Übersicht, was sie überhaupt alles anbieten und in welchen Genres die Spiele eingeordnet sind zur Erinnerung. Zusätzlich zum Überblick über die Konkurrenz in gleicher Weise. Wie sind die Zahlen der Spiele im Detail in den verschiedenen Märkten? Wie sind die Zahlen der Spiele des Genres im Vergleich zur Konkurrenz in dem Genre? Wo liegt das Verbesserungspotenzial des Publishers in seiner Ausrichtung der Spiele? Wo gibt es Marktlücken und wo liegen die Stärken? 

%Beschreiben sie die Personengruppe oder Personengruppen, die das von ihnen benannte Anwendungsproblem lösen möchte. Auf welches Vorwissen können sie in dieser Gruppen von Anwenderinnen aufbauen? Welche Informations"-bedürf"-nisse werden durch die Visualisierungen adressiert?

\subsection{Überblick und Beiträge}
Überblick Daten: XBoxOne Datensätze aus 2016 mit den Attributen Position in der Tabelle, Publisher, Jahr der Veröffentlichung, Genre, Verkaufszahlen global, VK NA, VK EU, VK Asien, VK andere von der Platform Kaggel. Info Datenherkunft auf Kaggle hinzufügen. relevant: alle außer Jahr der Veröffentlichung
Überblick Visualsierungen:  
Visualisierung 1: explizites Baumdiagramm zur Übersicht über Publisher und den von ihnen jeweils bedienten Genres und den Spielen in den Genres jeweils zur Beantwortung Fragengruppe 1
Visualierung 2: Scatterplot zur Beantwortung von Fragegruppe 2, der tieferen Marktanalyse, also wie sich die Spiele in einem Genre im Vergleich verkauften mit Zusammenhängen zwischen globalen Verkäufen und Verkäufen in den auswählbaren Regionen. So kann ein Publisher vergleichen, wie seine Spiele eines Genres im Vergelich zur Konkurrenz verkauft wurden bzw wie erfolgreich seine Spiele waren.
Visualisierung 3: parallele Koordinaten zur Beantwortung von Fragegruppe 3, damit Publisher auf einen Blick vergleichen können, wie ein konkretes Spiel von ihnen in den verschienden Regionen/Märkten verkauft werden konnte. Auf welchen Märkten war es am erfolgreichsten, welche Märkte sollten stärker beworben werden, auf welche sollte der Fokus gelegt werden? Spiele auswählbar je nach Genre.
Beitrag allgemein der Visualisierungen: Lösung Problemstellung und Fragegruppe 4, also der allgemeinen. schnellerer Überlick über Erfolg der Spiele des Publishers in den Märkten und in Konkurrenz zu den anderen Spielen eines Genres zur zukünftigen Ausrichtung des Publishers.
kurze Vorstellung des Vorgehens im Projektbericht: 1. Daten, 2. Visualisierungsbeschreibung + Interaktion, 3. Konkrete Umsetzung/Implementierung, 4. Konkreter Anwendungsfall, also ein Publisher aussuchen und durch die Visualisierungen gehen und dabei die oben beschriebenen Fragen beantworten und Problem lösen, 5. Verwandte Arbeiten vorstellen am besten in Kombi BWL und Visual Analytics und Spielemarkt, 6. Zusammenfassung und ausblick. Für Ausblick kann integriert werden, dass die analyse erweritert werden kann mit adäquaten intern erhobnen Daten der Publisher zu Umsätzen und Gewinnen insgesamt und pro Spiel + zu Kosten pro Spiel für die Entwicklung/Marktreife + Kosten Spiel auf dem Mart zur tieferen Analyse der Verkaufszahlen


%In diesem Abschnitt geben sie einen kurzen Überblick über die Daten und verwendeten Visualisierungen. Dann benennen sie die Beiträge ihres Projekts. Diese Beiträge müssen sie in den hinteren Teilen des Berichts genauer ausführen und belegen.

\section{Daten}
Der für das vorliegende Visualisierungsprojekt verwendete Datensatz stammt von der Plattform \textit{"Kaggle"} von dem Nutzer \textit{"SID_TWR"} mit dem Titel \textit{"Video Games Sales Dataset"}. Die Daten entstanden laut Nutzer durch Erweiterung der Daten eines Web Scrapes von \textit{"VGChartz Video Games Sales"} motiviert durch Gregpry Smith um weitere Attribute aus einem Web Scrape von \textit{"Metacrtitic"}.  %hier einfügen Link/Bibliografie zu Kaggle & zu VG Chartz 

Der Nutzer stellt drei Datensätze zur Verfügung, von denen für dieses Projekt der Datensatz \textit{"XBoxOne_GameSales"} ausgewählt wurde. Hierbei ist zu beachten, dass die zuvor erwähnten Erweiterungen um Attribute stammend von \textit{"Metacrtitics"} in diesem Datensatz nicht enthalten sind. 
Der Originaldatensatz liegt als CSV-Datei vor und beinhaltet zehn Spalten mit 613 einzelnen Positionen, also Videospielen. 
Zunächst ist eine eindeutige Identifikationsnummern gegeben, hier als \textit{"Positionen"} bezeichnet. Weiterhin sind der Videospielnamen sowie das jeweilige Jahr der Veröffentlichung der Spiele aufgelistet. 
Das Attribut \textit{"Genre"} kategorisiert die Videospiele in verschiedene Genre und \textit{"Publisher"} ordnet jedem Videospiel seinen Verleger zu. 
Zuletzt sind die Attribute \textit{"North America"}, \textit{"Europe"}, \textit{"Japan"}, \textit{"Rest of World"} und \textit{"Global"} aufgeführt. Sie stellen die Verkäufe der Videospiele in den Regeionen Nordamerika, Europa, Japan, dem Rest der Welt und auf globaler Ebene dar. 
Die Einheit ist hier bei allen \textit{"millions of units"}, also Millionen Stück verkaufter Kopien, was durch eine Recherche auf VGChartz Video Games Sales deutlich wird. %hier Link/Bibliografie zu VGChartz
Hochgeladen wurden die Daten circa vor drei Jahren. Der für das Projekt ausgewählte Datensatz bildet Videospiele von 2013 bis ungefähr 2020 ab, wobei seit dem Veröffentlichungsjahr 2019 keine Verkaufszahlen mehr eingetragen sind. 
Auch auf \textit{"VGChartz Video Games Sales"} finden sich schon im Laufe des Veröffentlichungsjahres 2018 weniger bis keine Verkaufszahlen. Zwar wurden die Videospiele bis heute weiter eingetragen, jedoch ohne Aktualisierung der Verkaufszahlen. 
So ist davon auszugehen, dass die für das Projekt verwendeten Daten Verkaufszahlen ab 2013 bis circa 2018 abbilden. Der Stand der Zahlen hat sich auch laut \textit{"VGChartz Video Games Sales"} aufgrund unbekannter Gründe nicht geändert. 
Es werden die kumulierten Verkaufszahlen in Millionen Stück in den Jahren seit Erscheinungsdatum bis circa Ende 2018 dargestellt. 

Insgesamt werden die vorhandenen Daten als gut geeignet für die anvisierte Zielgruppe des oberen und mittleren Managements der Verleger von Videospielen für die Platform XBox One und das zuvor eingeleitete Zielproblem eingeschätzt. 
Die Daten ermöglichen eine grobe Übersicht über die verschiedenen Verleger und ihre Angebote in Form von Videospielen und bedienten Genres. 
Weiterhin schlüsseln sie detailliert auf, wie sich ein Videospiel seit seiner Veröffentlichung global, aber auch in den einzelnen Weltregionen verkaufte. Auf Basis dieser Daten ist eine Marktanalyse zur Lösung des Problems der fehlenden Übersicht und Kenntnis der Leistung der Videospiele eines Publishers am Markt sowie seiner direkten Konkurrenz im Genre möglich. 
Denkbar ist jedoch für eine weitergehende und tiefere Analyse die Einbeziehung von Daten zu Kritiken, Entwicklungskosten, Kosten pro Videospielkopie am Markt sowie die durch den Verkauf eines Spiels erzielten Umsätze. 
Diese Daten sind jedoch größtenteils nicht öffentlich zugänglich, unvollständig oder müssten mit sehr hohem Zeitaufwand aus diversen Quellen zusammengetragen werden. 
Die Visualisierung dieser Daten würde das hier gesetze Zielproblem ergänzen, jedoch die anvisierte überblicksartige Marktanalyse überschreiten. In einer unabhängigen, weiteren Martkanalyse und den zugehörigen Visualisierungen wäre dies jedoch denkbar.    

Der Datensatz enthält jeodch auch fehlende Werte bei den Verlegern, fortan Publisher genannt, und den Verkaufszahlen jeder Region. Die Daten werden also um jene Videospiele bereinigt, denen kein Publisher zugeordnet ist oder deren Publisher unbekannt ist. 
Weiterhin werden jene Videospieldaten gelöscht, deren Verkäufe in allen Regionen bzw. global einen Wert von Null aufweisen. 
Zuletzt wird die Zahl der Videospiele um jene reduziert, deren Publisher nur ein Spiel herausgibt.  
Die Spalte des Erscheinungsjahres des jeweiligen Videospiels wird aus der hier durchgeführten Marktanalyse und ihrer Visualsierung ausgegliedert. 
Hieraus können weder Daten für ein Zeitreihendiagramm extrahiert werden, noch macht ein spezifischer Vergleich der Verkaufszahlen in Relation zum Erscheinungsjahr für die Problemstellung und deren Lösung in einem ersten Überblick Sinn. Eine Analyse dieser Daten gliedert sich in mögliche weiterführende Visualisierungen, wie im obigen Abschnitt beschrieben, ein. 

\subsection{Bereitstellung der Daten}
Wie zuvor beschrieben, werden die originalen Daten über \textit{Kaggle} zur Verfügung gestellt. Diese werden als CSV-Datei heruntergeladen und zusätzlich zur Bearbeitung in das ODS-Format überführt. 
Alle Daten sowie das Projekt und dieser Bericht werden in einem öffentlichen GitHub Repository bereitgestellt. 
Die Datendateien sind im Ordner \textit{Daten} in den Unterordnern \textit{CSV}, \textit{JSON} und \textit{Tabelle} je nach Format zu finden. Die Dateien mit den Originaldaten enthalten die Endung \textit{_Original}, Dateien mit Testdaten die Endung {_test} und finale für das Projekt nutzbare Dateien die Endung \textit{_Projekt}.
Die in Tabellenform überführten Originaldaten werden im ODS-Format bearbeitet, gelöscht, in einer neuen Tabellendatei gespeichert und wieder in CSV konvertiert. Die Datenvorverarbeitung im Detail ist im folgenden Unterkapitel einsehbar. 
Die CSV-Datei mit der Endung \textit{_JSON} bildet die Grundlage für die Erstellung der JSON-Datei.

Um die Visualisierung mittels explizitem Baumdiagramm zu erstellen, wird eine JSON-Datei benötigt, in der die Beziehungen der Publisher, Genre und Videospiele zueinander beschrieben sind. 
Videospiele sind hier Kinder der Genres, die wiederum Kinder der Publisher sind und unter einem Wurzelknoten zusammengefasst werden. Für jeden Publisher sind die Genres als Kinder beschrieben, die er bedient. Unter jedem Genres, das ein Publisher bedient werden dann die Videospiele kategorisiert. 
Die Datei findet sich im Ordner \textit{Daten/JSON} mit der Bezeichung \textit{XBoxOne_GamesSales_Projekt.json}.

Für die Realisierung der Visualisierungstechniken des Scatterplots und der parallelen Koordinaten wird dieselbe CSV-Datei benötigt. Sie ist im Ordner \textit{Daten/CSV} mit der Bezeichnung \textit{XBoxOne_GamesSales_Projekt} zu finden. 
In dieser Datei sind die nachfolgend detailliert beschriebenen Modifikationen enthalten. 
Sie beinhaltet entsprechend die alle Positionen, Videospielnamen, Genres, Publisher, Verkäufe in Nordamerika, Europa, Japan, dem Rest der Welt und die globalen Verkäufe. 
Dezimaltrennzeichen sind als Punkt realisiert und die einzelnen Werte in der Datei sind mit Kommata voneinander getrennt. 

\subsection{Datenvorverarbeitung}
Zur besseren Nachvollziehbarkeit der Datenvorberarbeitung wird im GitHub Repository im Ordner \textit{Daten} eine zweite README.md-Datei erstellt, die den Fortlauf der Vorverarbeitung dokumentiert. 
Da der originale Datensatz einige Anpassungen verlangt, um für die Visualisierung sinnvoll nutzbar zu sein, ist eine Datenvorberarbeitung und Modifikation unausweichlich. Die benötigte Datengrundlage wird in sieben Schritten erreicht.

Zuerst werden die Daten als CSV-Datei heruntergeladen und unter der Bezeichnung \textit{XBoxOne_GamesSales_Original.csv} im Ordner \textit{Daten/CSV} gespeichert. 
Außerdem wird die CSV-Datei in ein ODS-Format konvertiert, um mit dem Programm \textit{Open Office Calc} besser bearbeitet werden zu können. Dies Datei ist unter \textit{XBoxOne_GamesSales_Original.ods} im Ordner \textit{Daten/Tabelle} zu finden. 

In einem zweiten Schritt werden Testversionen der Originaldaten mit den ersten 20 Positionen erstellt im CSV- und ODS-Format erstellt. Mit ihnen können die Visualisierungen zunächst übersichtlich auf Funktionaltität getestet werden. 
Erst in einem späteren Schritt werden dann nur die Quellen für die Daten im Programm ausgetauscht. Die Testdaten werden zudem um die Spalte des Erscheinungsjahres reduziert. Eine Erklärung folgt in Schritt drei.
Die Testdateien werden unter \textit{XBoxOne_GamesSales_test.csv} und \textit{XBoxOne_GamesSales_Original.ods} in den Ordnern \textit{Daten/CSV} und \textit{Daten/Tabelle} gespeichert.

Im dritten Schritt erfolgt die eigentliche Modifikation in den Originaldaten in originaler Größe. Die Originaldatei im ODS-Format wird zusätzlich als \textit{XBoxOne_GamesSales_Projekt.csv} im Ordner \textit{Daten/Tabelle} gespeichert, sodass hier ohne Veränderung der Originaldaten modifiziert werden kann.
Zunächst wird die Spalte  \textit{Year} gelöscht. Diese Daten sind für die Visualisierungen nicht nötig. Die Daten geben keine Grundlage für eine Visualisierung als Zeitreihe. 
Das Erscheinungsjahr könnte jedoch in Relation zu den Verkaufszahlen betrachtet werden, da diese im ersten Moment durch die Kumulation der Verkaufszahlen über die Jahre größer sein können, je länger das Spiel auf dem Markt ist. 
Jedoch kann auch ein bspw. erst 2017 erschienenes Videospiel höhere Verkaufszahlen aufweisen als eines, das bspw. im Jahr 2013. Zuverlässige, eindeutige und schnell erkennbare Schlüsse, die für eine hier angestrebte überblicksartige Marktanalyse nötig sind, können somit nicht ermöglicht werden. 
Es werden zusätzlich jene Positionen, also Videospiele, eliminiert, die einen Wert von Null in allen Regionen der Verkäufe enthalten. 
Auch unter Anbetracht der Tatsache, dass hier zumindest geringe Verkäufe unter 100.000 verkauften Stück erzielt worden sein könnten, ist ein Vergleich und eine Visualisierung sinnlos.
Weiterhin werden die Positionen gelöscht, die keinen Publisher oder den Wert  \textit{Unknown} an dieser Stelle enthalten. Sie sind für die Zielgruppe und das Zielproblem irrelevant.
Zusätzlich wird der Datensatz um die Positionenen minimiert, die von einem Publisher stammen, der nur dieses eine Videospiel verlegt. Ein Vergleich der Spiele des Publishers untereinander und mit Konkurrenten in dem Genre ist nicht sinnvoll. 
Zudem ist damit zu rechnen, dass sich jene Publisher sehr spezifisch ausgerichtet haben und keine hier visualisierte überischtsartige Marktanalyse benötigen.
Die entstehende Datei wird zuletzt wieder in das CSV-Format konvertiert und unter \textit{XBoxOne_GamesSales_Projekt.csv} im Ordner \textit{Daten/CSV} gespeichert.

Im vierten Schritt wird der Name des Publishers \textit{Namco Bandai Games} in \textit{Bandai Namco Games} vereinheitlicht, da beide denselben Verleger darstellen. Dieser wurde 2014 in letzteren umbenannt

Als Vorbereitung für die Erstellung der JSON-Datei wird im fünften Schritt eine weitere CSV-Datei erstellt, die die Abhängigkeiten für die gewünschte Baumstruktur als Spaltenüberschriften enthält. 
Die benötigten Daten des Publishers, Genres und Videospielnamens werden entsprechend der Struktur übertragen. Die Datei ist unter \textit{XBoxOne_GamesSales_JSON.csv} im Ordner \textit{Daten/CSV} zu finden.

Der fünfte Schritt beinhaltet die Konvertiertung der erstellten CSV-Datei mit den Abhängigkeiten in eine JSON-Datei. Diese wurde mittels des Online-Tools \textit{convertcsv.com} vorgenommen. %hier Link/Bibliografie dazu
Der entstandene Code wird im Editor \textit{Visual Studio Code} in eine neue Datei kopiert und zunächst unter \textit{Daten/JSON} gespeichert.

Zuletzt wird der JSON-Code auf Fehler in den Abhängigkeiten konrolliert und verbessert. Dasselbe wird respektive in der Datei \textit{XBoxOne_GamesSales_JSOn.csv} vorgenommen. 
In der JSON-Datei wird ein Wurzelknoten hinzugefügt und die leeren Felder, die im expliziten Baumdiagramm leere Blattknoten darstellen würden, gelöscht.

Für das in diesem Projekt anvisierte Ziel und den beschriebenen Anwendungsfall sind weitere Berechnungen nicht nötig. %eventuell falls sinnvoll Durchschnittswerte in Elm der globalen Verkaufszahlen in einem Genre berechnen für Scatterplot
Desweiteren ist eine Umrechnung der Einheiten der Verkäufe nicht ratsam, da sie in der vorliegenden Form sehr übersichtlich sind. 
Bei einer Umrechnung in bspw. Tausend Stück würde die Zahl schon zu unübversichtlich und nicht auf den ersten Blick erkennbar sein, wie es für die Zielgruppe notwendig ist. Zur Erläuterung der Einheiten wird ein Informationstext erstellt und die Achsenbeschriftungen entsprechend angepasst.

Zunächst wurde eine Filterung der Daten auf Null-Werte in den Verkäufen jeder Region in Betracht gezogen, sodass diese Videospiele, sollten sie auch nur in einer Region einen Null-Wert enthalten, nicht angezeigt würden. 
Dies ergibt jedoch wenig Sinn im Zusammenhang einer übersichtsartigen Marktanalyse. Hierfür sind auch Informationen über keine erfolgten Verkäufe und die Region, in der nichts verkauft werden konnte, wichtig. 
Zu beachten ist jedoch, dass als Null-Werte auch solche aufgeführt sind, deren Verkaufszahlen bei unter 0.01 Millionen Stück verkaufter Videospielkopien liegt. In Anbetracht der Größe der Märkte sind diese sehr geringen Verkaufszahlen jedoch als Null und entsprechend verbesserungswürdig anzusehen.
Entsprechende Erklärtexte sind notwendig.

%Beschreiben Sie vorhandenen Daten. Gehen sie kritisch darauf ein, in wie weit sich die Daten für die Bearbeitung der Fragestellungen und dem Erreichen von Lösungen für die oben beschriebene Zielgruppen eignen. Haben sie die Daten sinnvoll mit weiteren Datenquellen ergänzt? Wenn ja, wie?
%Erklären sie die technische Bereitstellung der Daten.
%Wie sind die Daten zugänglich? Welche Formate werden genutzt. Gibt es Besonderheiten beim Lesen der Formate?
%Beschreiben sie die Datenvorverarbeitung.
%Welche Datenvorverarbeitungsschritte sind notwendig? Beschreiben Sie die einzelnen Schritte und begründen sie sie, z.B. warum werden manche Daten weggelassen, über welche Mengen werden Durchschnitte berechnet, warum sind die so berechneten Werte aussagekräftiger als andere Werte. Wenn möglich sollen sie die Datenvorverarbeitung in Elm programmieren, so dass ihre Anwendung auf eine Änderung der Rohdaten reagieren kan. 

\section{Visualisierungen}
\subsection{Analyse der Anwendungsaufgaben}
Analysieren sie die konkreten Anwendungsaufgaben, die die Lösung des Zielproblems durch die Anwender:innen bearbeitet werden müssen. 
Welche sinnvollen mentale Modelle helfen den Personen bei der Bearbeitung. 
%Welche Visualisierungen helfen den Personen, die die Software verwenden, sinnvolle mentale Modelle aufzubauen. 
Sind diese mentalen Modelle für sie notwendig, um die Aufgaben lösen zu können? Gehen sie bei ihrer Argumentation von den Anwendungsaufgaben aus und kommen sie dann zu den mentalen Modellen, deren Aufbau durch Visualisierungen unterstützt wird. 
\subsection{Anforderungen an die Visualisierungen}
Leiten sie Anforderungen an das Design der Visualisierungen ab, die sich durch ihre Analyse des Zielproblems ergeben.
\subsection{Präsentation der Visualisierungen}
Präsentieren sie die visuelle Abbildungen und Kodierungen der Daten und Interaktionsmöglichkeiten. 
Sie müssen  begründen, warum und wie gut ihre Designentscheidungen die erstellten Anforderungen erfüllen. 
Weiterhin müssen sie begründen, warum die gewählte visuelle Kodierung der Daten für das zulösenden Problem passend ist.
Typische Argumente würden hier auf Wahrnehmungsprinzipien und Theorie über Informationsvisualisierung verweisen. 
Die besten Begründungen diskutieren explizit die konkrete Auswahl der Visualisierungen im Kontext von mehreren verschiedenen Alternativen. 
Machen sie hier nicht den Fehler, einfach nur Visualisierung aus den vorgegebenen Bereichen zu diskutieren, weil das in der Regel nicht sinnvoll ist.
Wenn sie sich für einen Scatterplot entschieden haben, ist ein Zeitreihendiagramm in der Regel keine Alternative.
Diskutieren sie also nicht einfach Zeitreihendiagramme, weil sie in den Anforderungenen an das Projekt neben Scatterplots stehen, sondern suchen sie nach echten alternativen Visualisierungen, die zum Aufbau eines vergleichbaren mentalen Modells führen. 
Diskutieren sie die Expressivität und die Effektivität der einzelnen Visualisierungen. 

Die eben beschriebenen Präsentationen und Begründungen sollen für jede der drei folgenden Visualisierungen durchgeführt werden. 
\subsubsection{Visualisierung Eins}
\subsubsection{Visualisierung Zwei}
\subsubsection{Visualisierung Drei}

\subsection{Interaktion}
Die präsentierten Visualisierungstechniken müssen interaktiv zu einer Anwendung verknüpft werden.
Die Interaktion mit einer Visualisierung soll in den anderen Visualisierungen zu einer Änderung führen. 
Erklären sie die möglichen Interaktionen mit den einzelnen Visualisierungen und die möglichen Verknüpfungen zwischen ihnen. Begründen Sie warum die konkreten Interaktionen umgesetzt wurden und welche Zwecke für die Anwenderinnen mit ihnen unterstützt werden. Begründen sie ebenfalls warum sie andere Interaktionsmöglichkeiten nicht umgesetzt haben. Wenn sie keine der geforderten Interaktionen umsetzen, erhalten Sie im gesamten Projekt deutlichen Punktabzug. 

\section{Implementierung}
Beschreiben Sie die Implementierung ihrer Visualisierungsanwendung in Elm. Stellen die Gliederung ihres Quellcodes vor. Haben Sie verschiedene Elm-Module erstellt. Was war aufwändig umzusetzen, was ließ sich mit dem vorhanden Code aus den Übungen relativ einfach umsetzen? 

Wie sieht die Elm-Datenstruktur für das Model aus, in dem die verschiedenen Zustände der Interaktion gespeichert werden können.

\section{Anwendungsfälle}
Präsentieren sie für jede der drei Visualisierungen einen sinnvollen Anwendungsfall in dem ein bestimmter Fakt, ein Muster oder die Abwesenheit eines Musters visuell festgestellt wird. Begründen sie warum dieser Anwendungsfall wichtig für die Zielgruppe der Anwenderinnen ist. Diskutieren sie weiterhin, ob die oben beschriebene Information auch mit anderen Visualisierungstechniken hätte gefunden werden können. Falls dies möglich wäre, vergleichen sie die den Aufwand und die Schwierigkeiten ihres Ansatzes und der Alternativen. 
\subsection{Anwendung Visualisierung Eins}
\subsection{Anwendung Visualisierung Zwei}
\subsection{Anwendung Visualisierung Drei}

\section{Verwandte Arbeiten}
Führen sie eine kurze Literatursuche in der wissenschaftlichen Literatur zu Informationsvisualisierung und Visual Analytics nach ähnlichen Anwendungen durch. Diskutieren sie mindestens zwei Artikel. Stellen sie Gemeinsamkeiten und Unterschiede dar.

\section{Zusammenfassung und Ausblick}
Fassen sie die Beiträge ihre Visualisierungsanwendung zusammen. Wo bietet sie für die Personen der Zielgruppe einen echten Mehrwert.

Was wären mögliche sinnvolle Erweiterungen, entweder auf der Ebene der Visualisierungen und/oder auf der Datenebene?

\section*{Anhang: Git-Historie}

\printbibliography

\end{document}

